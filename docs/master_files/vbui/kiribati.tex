\documentclass{article}
\author{
    Epidemiological Modelling Unit,
    \\ School of Public Health and Preventive Medicine,
    \\ Monash University
}
\usepackage{graphicx}
\usepackage{biblatex}
\usepackage{threeparttable}
\usepackage{tabularx}
\usepackage{alphabeta}
\usepackage{amsmath}
\usepackage{caption} \captionsetup[table]{singlelinecheck=false}
\usepackage[labelfont=bf]{caption}
\usepackage[left=2.5cm,right=2.5cm]{geometry}
\bibliography{../../references/emu_library}

\title{Modelling methods, Kiribati application}

\begin{document}

\maketitle

\begin{document}

\section{Model Structure}

\end{document}
\begin{figure}[ht]
    \includegraphics[width=\textwidth]{../../tex_descriptions/models/tuberculosis/model.png}
    \caption[Unstratified compartmental model structure.]{Unstratified compartmental model structure. \small S = susceptible, E: Early LTBI, L: Late LTBI, E = exposed, I = active, R = recovered/removed}
    \label{fig:seeiir}
\end{figure}
\section{Model stratification}
\section{Age stratification}
% Note that this will vary for every application, so will need to be edited - not sure of how best to manage this:
All compartments of the base compartmental structure were stratified by age into the following age bands:
\begin{itemize}
    \item 0 to 4 years
    \item 5 to 9 years
    \item 10 to 14 years
    \item 15 to 19 years
    \item 20 to 24 years
    \item 25 to 29 years
    \item 30 to 34 years
    \item 35 to 39 years
    \item 40 to 44 years
    \item 45 to 49 years
    \item 50 to 54 years
    \item 55 to 59 years
    \item 60 to 64 years
    \item 65 to 69 years
    \item 70 to 74 years
    \item 75 years and above
\end{itemize}
Demographic processes, including births, ageing and non-infection-related deaths 
are not simulated, given the timeframes considered in this simulation.

\subsection{Births and deaths}
Births are modelled using time-variant crude birth rates that are multiplied by the modelled population 
size to determine the number of newborn individuals entering the model at each time. A time-variant 
and age-specific rate or non-TB-related mortality applies to all model compartments to simulate 
deaths from other causes than TB. We use estimates from the UN population division to inform the 
birth and mortality rates.
We also apply additional death rates to the compartments I and T to reflect mortality induced by TB 
disease.
\subsection{M.tb Tranmission}
We use different levels of susceptibility to infection for individuals who are currently latently infected 
with M.tb or have recovered from active TB, as compared to infection-naive individuals. The effect of 
BCG vaccination is captured by reducing the susceptibility to infection of individuals under the age of 
30 years old. We assume a 70\% reduction in the susceptibility of BCG-vaccinated children under the 
age of 15 years old. A linear function is used to reflect the progressive loss of BCG immunity 
between the age of 15 and 30 years old. 
\section{Infectious seed}
The infectious seed value is assigned at the start of simulation process.
This value is subtracted from the total modelled population and assigned to the susceptible compartment, while all other compartments are assigned a starting value of zero.
This process is undertaken before age stratification is applied,
with the stratification process then splitting these values proportionately according to the starting age distribution of the population.

\section{Progression Parameters}
\subsection{Progression from latent to active TB}
We use the estimates reported in Ragonnet et al. to inform the modelled dynamics of activation from 
latent to active TB. These parameters vary by age, and a multiplier is used to 
incorporate uncertainty around the progression rates
\subsection{Effect of diabetes}
The model is not stratified by diabetes status. Instead, we model the effect of diabetes type 2 by 
increasing the rates of progression from latent to active TB using age-specific multipliers. For each 
age group, the value of the diabetes-effect multiplier depends on the age-specific proportion of 
diabetic individuals and the relative rate of TB reactivation for diabetic individuals compared to non-diabetic individuals.
\subsection{Natural history flows}
We use the estimates reported in Ragonnet et al. to model the rate of TB mortality in the absence of 
treatment and the rate of self-recovery. We use different rates of untreated TB mortality and self-recovery for smear-positive TB compared to smear-negative TB. The TB mortality and self-recovery 
rates associated with extrapulmonary TB are assumed to be the same as those of smear-negative TB.
\subsection{Passive case detection of active TB}
The detection rate is defined as the rate of progression from the active disease to the treatment 
compartment, as all detected individuals are assumed to be started on treatment at diagnosis in our 
model. This rate is calculated by multiplying the screening rate with the diagnostic test sensitivity. 
The screening rate can be interpreted as the reciprocal of the average time that diseased individuals 
take to seek care. The diagnostic sensitivity varies according to the organ status to reflect the relative 
differences in the difficulty to diagnose smear-negative TB and extrapulmonary TB, as compared to 
smear-positive TB.
We use a time-variant function to model the screening rate in order to capture detection improvements 
over time.
\subsection{Treatment outcomes}
Treated individuals can experience three different treatment outcomes: treatment success, relapse or 
death. The rate of treatment-induced recovery \phi{} is set to the reciprocal of the duration of a 
completed treatment course. We then use the observed treatment success proportion (often referred to 
as “treatment success rate”) as model input. In our model, it is calculated from \(TSR = \frac{\phi}{\phi + \rho + \mu_{\tau} + \mu^{'}}\){}, 
where \rho{} is the relapse rate, \({\mu_{\tau}{}}\) is the excess mortality rate of individuals on TB treatment, and \mu is the 
non-TB-related mortality rate. Finally, we calculate the respective values of {\rho}{} and \({\mu_{\tau}{}}\) using the 
observed proportion of deaths among all negative treatment outcomes, denoted \pi{}. We have \({\pi = \frac{\mu_{\tau} +\mu}{\rho + \mu_{\tau} + \mu}}\){} that we inject into the \({TSR}\) equation.


\section{Modelled Intervention}
\subsection{Population-based screening and treatment of LTBI}
Mass LTBI screening and treatment is implemented as part of the intervention conducted in South Tarawa in 
2022. This is modelled by making latently infected individuals (from E and L) transition to the 
recovered compartment (R). The rate associated with these flows is obtained by multiplying the LTBI 
screening rate with the sensitivity of the LTBI test employed and the individual-level efficacy of 
preventive treatment. The LTBI screening rate is implemented as a time-variant parameter that is 
stratified by location
\subsection{Active case finding}
Active case finding (ACF) is implemented to simulate the interventions linked to the detection of 
individuals with active TB implemented in South Tarawa. This is modelled by 
implementing an additional transition flow from compartment I to compartment T. The rate associated 
with this flow is obtained by multiplying the location-specific ACF screening rate with the sensitivity 
of the detection algorithm used for the ACF intervention. The ACF screening rate is implemented as a 
time-variant parameter.
\subsection{Calculation of the screening rates}
To simulate the interventions, we apply a positive rate of ACF and/or LTBI screening over the 
intervention periods. The screening rates are determined such that the modelled total proportion of the 
population screened corresponds to the true population proportion screened. The screening rate is set 
equal to \(-log(1 - coverage)\) for the year during which the intervention is implemented, where 
\(coverage\) is the total proportion of the population screened by the intervention. 
\section{Parameters}
\subsection{Target Parameters}
\begin{table}[!ht]
    \caption {Parameters} 
    \centering
    \begin{tabular}{ |l|c| }
        \hline
            Parameter  & Range  \\ \hline
            Initial population size  & 200 - 800  \\ \hline
            Transmission scaling factor  & 0.002 - 0.01  \\ \hline
            Progression multiplier  & 0.5 - 2.0  \\ \hline
            Screening profile (inflection time), year  & 2000.0 - 2020.0  \\ \hline
            Screening profile (shape)  & 0.07 - 0.1  \\ \hline
            Screening profile (final rate), per year  & 0.4 - 0.55  \\ \hline
            Relative rate of passive TB screening in Ebeye (ref. Majuro)  & 1.3 - 2.0  \\ \hline
            Relative rate of passive TB screening in other islands (ref. Majuro)  & 0.5 - 1.5  \\ \hline
            Relative rate of TB progression for diabetic individuals  & 2.0 - 5.0  \\ \hline
            Relative risk of infection for individuals with latent infection (ref. Infection-naive)  & 0.2 - 0.5  \\ \hline
            Relative risk of infection for individuals with history of infection (ref. Infection-naive)  & 0.2 - 1.0  \\ \hline
            Efficacy of preventive treatment  & 0.75 - 0.85  \\ \hline
            Relative screening rate following ACF interventions (ref. Before intervention)  & 1.0 - 1.5  \\ \hline
            TB mortality (smear-positive), per year  & 0.335 - 0.449  \\ \hline
            TB mortality (smear-negative), per year  & 0.017 - 0.035  \\ \hline
            Self-cure rate (smear-positive), per year  & 0.177 - 0.288  \\ \hline
            Self-cure rate (smear-negative), per year  & 0.073 - 0.209  \\ \hline
        \end{tabular}
    \end{table}

    \begin{table}[!ht]
        \caption {Baseline inputs}  
        \centering
        \begin{tabular}{|l|l|l|}
        \hline
            Variable & Targeted value & Source \\ \hline
            TB prevalence in Kiribati  & ………………………………. & ~ \\ \hline
            TB notifications in Kiribati  & ~ & Literature \\ \hline
            ·         2003 & ·         400 & ~ \\ \hline
            ·         2004 & ·         428 & ~ \\ \hline
            ·         2005 & ·         450 & ~ \\ \hline
            ·         2006 & ·         501 & ~ \\ \hline
            ·         2007 & ·         433 & ~ \\ \hline
            ·         2008 & ·         424 & ~ \\ \hline
            ·         2009 & ·         344 & ~ \\ \hline
            ·         2010 & ·         347 & ~ \\ \hline
            ·         2011 & ·         409 & ~ \\ \hline
            ·         2012 & ·         407 & ~ \\ \hline
            ·         2013 & ·         472 & ~ \\ \hline
            ·         2014 & ·         473 & ~ \\ \hline
            ·         2015 & ·         559 & ~ \\ \hline
            ·         2016 & ·         575 & ~ \\ \hline
            ·         2017 & ·         422 & ~ \\ \hline
            ·         2018 & ·         349 & ~ \\ \hline
            ·         2019 & ·         436 & ~ \\ \hline
            ·         2020 & ·         425 & ~ \\ \hline
            Total population size in 2020 & 120000 & 2020 National Census \\ \hline
        \end{tabular}
    \end{table}
    
\newpage
\printbibliography

\end{document}
