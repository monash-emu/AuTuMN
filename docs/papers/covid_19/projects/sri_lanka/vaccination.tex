\subsection{Modelling vaccine effects}
We stratified all model compartments as either vaccinated or unvaccinated and commenced simulations with an fully unvaccinated population. With vaccination roll-out, individuals in the susceptible and recovered compartments move from “unvaccinated” to “vaccinated” at a constant rate representing vaccine administration over time. The daily rate of vaccination is calculated from the vaccine coverage achieved over a given period of time is calculated as $rate_{vac} = \frac{-log(1-coverage)}{T}$, where coverage represents the targeted proportion of people vaccinated by the end of the roll-out period and \textit{T} represents the roll-out period.

Vaccination is assumed to have two mechanisms of effect: 1) prevention of infection and 2) protection against progressing to severe infection among those infected. A particular vaccine roll-out programme can be simulated to act through these two mechanisms simultaneously. The proportion of the effect that is attributed to preventing infection, $V_p = V_iV_e$ , where $V_i$ $\in [0,1]$ is the infection prevention efficacy and $V_e$ $\in [0,1]$ is the overall efficacy (that would be observed in clinical trials). If severity prevention efficacy is denoted $V_s$, since $V_e =V_i+V_s(1-V_i)$ it follows that $V_s =\frac {V_e(1-V_p)} {1- V_p V_e}$ . For the component of the vaccine effect attributed to infection prevention,
the infection risk of vaccinated individuals is reduced by ($1-V_i$). Severity-preventing vaccination reduces the infection fatality rate (IFR) and the probability that an infected individual experiences symptomatic disease. Thus, the vaccine efficacy parameter pertaining to disease severity prevention modifies the splitting proportions of infected individuals between the different clinical categories and the rate of COVID-19-related mortality.