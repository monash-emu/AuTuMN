\section{Calibration}

We calibrated the model using the adaptive Metropolis algorithm described by Haario et al. \cite{RN4}. A standard Metropolis algorithm with fixed proposal distribution parameters was used for the first 500 iterations to initiate the covariance matrix before the adaptive algorithm commenced. Seven chains were then run to ensure 10,000 iterations post-burn-in were achieved.

\subsection{Rationale for service-specific targets}
For all services (both metropolitan and regional), we included the time series of daily notifications for that service as a calibration target, using a normal distribution for the likelihood function. A normal distribution is preferred because the mapping process for the notifications for each service results in these quantities not being integer-valued.

In addition, we include time series for the following quantities at the state level. Because these quantities are counts, Poisson distributions are used in likelihood calculations:
\begin{itemize}
    \item Daily new COVID-19 notifications
    \item Daily new hospital admissions
    \item Daily new ICU admissions
    \item Daily deaths
\end{itemize}

\subsection{Assigning targets to services}
Hospital admissions and ICU admissions can be mapped directly to a health service cluster. Health service clusters include all health care (including public hospitals, private, rehab, acute, mental health, etc.) and some metropolitan services have changed service assignment over the years. Mapping was performed as at August 2020.  However, for the other two indicators used (notifications and deaths), mapping was not possible because these events do not necessarily occur within a health service cluster. Therefore, the local government area (LGA) of residence of the person notified or dying is considered. Each notification and death is split proportionately across the health service clusters to which they would typically present, according to historical data on hospital presentations for each LGA provided by DHHS. (Note that only notifications are considered as calibration targets, although these considerations are relevant to the comparison between data and modelled outputs undertaken for validation purposes.)

\subsection{Variation of age-specific proportion parameters using ``adjuster" parameters}
Our parameters included age-specific parameters that were varied up and down together during calibration. These proportion parameters are modified through ``adjuster" parameters that are not strictly multipliers, but are rather implemented in such a way as to scale the base parameter value while ensuring that the adjusted parameter remains a proportion (with range zero to one). In each of these cases, the adjuster parameters can be considered as multiplicative factors that are applied to the odds ratio that is equivalent to the baseline proportion to be adjusted. Specifically, the adjusted proportion is equal to:
\[\frac{proportion \times adjuster}{proportion \times (adjuster-1)+1}\]
This approach was applied for the age-specific fraction hospitalised and the age-specific fraction of active cases symptomatic.

\subsection{Variation of the proportion of patients symptomatic}
The modelled proportion of patients symptomatic differs by age group. However, given that this quantity remains highly uncertain and may vary between settings, it is varied during calibration. A single adjuster is used to increase or decrease each value for each age group.

\subsection{Variation of the proportion of patients hospitalised}
The modelled proportion of patients hospitalised similarly differs by age group, and is also likely to vary between settings. A single adjuster is used to increase or decrease each value for each age group.

\subsection{Variation of infection fatality rate}\label{vary_ifr}

The infection fatality rate can be defined as the risk of death given an episode of infection, including asymptomatic and undetected episodes. This is considered a more stable quantity than the case fatality rate. However, it is still likely to vary considerably between settings and so is adjusted during the calibration process. Because the epidemic in Victoria was characterised by high rates of transmission and disease in aged care, we fix the infection fatality rate for all age groups up to 74 years to the estimates derived from the literature, but vary the infection fatality rate for those aged 75 and above. This is intended to capture the increased risk of death for those in residential aged care during the second wave, the large majority of whom would be included in this age bracket.

\section{Likelihood function}

Likelihood functions are derived from comparing model outputs to target data at each time point nominated for calibration.

The composite likelihood function is given formally as:
\[\prod_{t}{n_{t}(\theta)d_{t}(\theta)h_{t}(\theta)i_{t}(\theta)} \times \prod_{t,g}{n_{t,g}(\theta,\sigma)}\]
where \(_{t}\) indexes the date, \(_{g}\) indexes the service, \(n_{t}\) refers to daily new notifications, \(d_{t}\) to daily deaths, \(h_{t}\) to daily new hospitalisations and \(i_{t}\) to daily new ICU admissions. The contributions of each state-wide component to the composite likelihood are measured with Poisson distributions (e.g. \(n_{t}(\theta)=Poiss(\nu_{t}(\theta))\), where \(\nu_{t}(\theta)\) is the number of notifications simulated by the model at date \(t\) under parameter set \(\theta\)), and normal distributions are used for each \(n_{t,g}\) (because these targets are not integer-valued). \(\sigma\) is the ratio of the peak of each service-specific notification to the corresponding standard deviation of each of the normal distributions used in calculating their contribution to the likelihood. This was included as a calibration parameter to improve calibration efficiency.