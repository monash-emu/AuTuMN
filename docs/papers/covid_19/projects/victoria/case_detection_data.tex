\subsection{Testing data}
Statewide daily testing data by date of test were provided by DHHS and applied to all health service clusters to provide a broad profile of the variation in testing capacity over time, including the lower testing numbers in early June compared to at the peak of the epidemic. Data sparseness precluded us from implementing separate functions for each individual health service cluster. For this application to Victoria, the case detection proportion corresponding to a per capita rate of testing of one test per thousand population per day was varied as a calibration parameter in creating the time-varying case detection proportion function. Note that testing rates were typically considerably higher than one per thousand per day during the period modelled, such that the actual modelled case detection proportion is considerably higher than the case detection calibration parameter for most of the simulation period.