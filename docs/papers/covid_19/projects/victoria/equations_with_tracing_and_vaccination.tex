
\section{Ordinary differential equations}

For the clearest description of the model, we refer the reader to our code repository, because our object-oriented approach to software development is intended to be highly transparent and readable. For those who prefer dynamical systems such as this presented in the form of ordinary differential equations, we present the following.

\[\frac{dS_{a,g,v=1}}{dt}= -(\lambda_{a,g,v=1}(t) \sigma_{a} + \zeta) S_{a,g,v=1} \]
\[\frac{dS_{a,g,v=2}}{dt}= -(\lambda_{a,g,v=2}(t) \sigma_{a} + \eta) S_{a,g,v=2} + \zeta S_{a,g,v=1} \]
\[\frac{dS_{a,g,v=3}}{dt}= -\lambda_{a,g,v=3}(t) \sigma_{a} S_{a,g,v=3} + \eta S_{a,g,v=2} \]
\[\frac{dE_{a,g,q=1,v}}{dt}=\lambda_{a,g,v}(t) \sigma_{a} S_{a,g,v} -\alpha E_{a,g,q=1,v} - \chi(t) E_{a,g,q=1,v} \]
\[\frac{dE_{a,g,q=2,v}}{dt}=-\alpha E_{a,g,q=2,v} + \chi(t) E_{a,g,q=1,v} \]
\[\frac{dP_{a,c,g,q,v}}{dt}=p_{a,c,v}(t) \alpha E_{a,g,q,v}-\nu P_{a,c,g,q,v}\]
\[\frac{dI_{a,c,g,q,v}}{dt}=\nu P_{a,c,g,q,v}-\gamma_{c}I_{a,c,g,q,v}\]
\[\frac{dL_{a,c,g,q,v}}{dt}=\gamma_{c}I_{a,c,g,q,v}-\delta_{a,c}L_{a,c,g,q,v}-\mu_{a,c}L_{a,c,g,q,v}\]
\[\frac{dR_{a,g,v=1}}{dt}=\sum_{c,q}{}\delta_{a,c}L_{a,c,g,q,v} -\zeta R_{a,g,v=1} \]
\[\frac{dR_{a,g,v=2}}{dt}=\sum_{c,q}{}\delta_{a,c}L_{a,c,g,q,v} +\zeta R_{a,g,v=1} - \eta R_{a,g,v=2} \]
\[\frac{dR_{a,g,v=3}}{dt}=\sum_{c,q}{}\delta_{a,c}L_{a,c,g,q,v} +\eta R_{a,g,v=2} \]

where
\[\lambda_{a,g,v}=\beta \sum_{g'}\textbf{G}_{g,g'} \sum_{j,c,v'} \psi_{v'} \frac{\epsilon P_{j,c,g',v'}(t)+\iota_{c}I_{j,c,g',v'}(t)+\kappa_{c}L_{j,c,g',v'}(t)}{N_{j,g',v'}(t)} C_{a,j}(t)\]

\[\sum_{c}p_{a,c,v}(t)=1,\forall t\in\mathbb{R}\]

\[\chi(t) = \frac{\alpha q(t) u(t)}{1 - q(t) u(t)}\]

\[\textbf{C}_{0}=\textbf{C}_{H}+\textbf{C}_{S}+\textbf{C}_{W}+\textbf{C}_{L}\]

\[\textbf{C}_{g}(t)=\textbf{C}_{H}+s_{g}(t)^{2}\textbf{C}_{S}+w_{g}(t)^{2}\textbf{C}_{W}+l_{g}(t)^{2}\textbf{C}_{L}\]

\[l_{g}(t)=\frac{re_{g}(t)+gr_{g}(t)+pa_{g}(t)+tr_{g}(t)}{4}\]

\begin{table}[ht]
\renewcommand{\baselinestretch}{1}
    	\begin{tabular}{| p{2cm} | p{11.1cm} |}
    	\hline
    		Symbol & Explanation \\
	    	\hline
	    	$S$ & Persons susceptible to infection \\
    		$E$ & Persons in the non-infectious incubation period \\
    		$P$ & Persons in the incubation period \\
    		$I$ & Persons in the early active disease period, before isolation or hospitalisation may occur \\
    		$L$ & Persons in the late active disease period, after isolation or hospitalisation may have occurred \\
    		$R$ & Persons in the recovered period, from which re-infection cannot occur \\
    		\hline
	\end{tabular}
\end{table}


\begin{table}[ht]
\renewcommand{\baselinestretch}{1}
    	\begin{tabular}{| p{2cm} | p{11.1cm} |}
    	\hline
    		Symbol & Explanation \\
    		\hline
    		\textit{t} & Time  \\
    		$_{\textit{a}}$ & Compartment of age group a \\
    		$_{\textit{c}}$ & Compartment of clinical stratification c \\
    		$_{\textit{g}}$ & Compartment of geographical service stratification g \\
    		$_{\textit{q}}$ & Compartment of tracing stratification q \\
    		$_{\textit{v}}$ & Compartment of vaccination stratification v \\
    		$\sigma$ & Susceptibility to infection \\
    		$\alpha$ & Rate of progression from non-infectious to infectious incubation period \\
    		$\nu$ & Rate of progression from infectious incubation to early active disease \\
    		$\gamma$ & Rate of progression from early active disease to late active disease \\
    		$\mu$ & Rate of disease-related death \\
    		$\epsilon$ & Relative infectiousness of pre-symptomatic compartment \\
    		$\iota$ & Clinical stratification infectiousness vector for early active compartment \\
    		$\kappa$ & Clinical stratification infectiousness vector for late active compartments \\
    		$\beta$ & Probability of infection per contact between an infectious and susceptible individual \\
    		$\psi$ & Relative infectiousness of vaccination status (unvaccinated, one dose or fully vaccinated) \\
    		$\zeta$ & Receipt of the first dose of a vaccination schedule \\
    		$\eta$ & Receipt of the second dose of a vaccination schedule \\
	    	\textit{j} & Infectious populations \\
    		\textit{p} & Proportion progressing to each clinical stratification \\
    		\textbf{G} & Square matrix of dimensions \(9 \times 9\) for nine services, as presented in Table \ref{tab:intercluster_mixing} \\
    \hline
	\end{tabular}
\end{table}

\begin{table}[ht]
\renewcommand{\baselinestretch}{1}
    	\begin{tabular}{| p{2cm} | p{11.1cm} |}
    	\hline
    	Symbol & Explanation \\
    	\hline
    	\textbf{C} & Mixing matrix \\
    	\textbf{H} & Household contribution to mixing matrix \\
    	\textbf{W} & Workplace contribution to mixing matrix \\
    	\textbf{O} & Other locations contribution to mixing matrix \\
    	\textbf{S} & Schools contribution to mixing matrix \\
    	\textit{l} & Other locations macrodistancing function of time \\
    	\textit{w} & Function fit to Google mobility data for workplaces \\
    	\textit{s} & Function fit to Google mobility data for schools \\
    	\textit{re} & Function fit to Google mobility data for retail and recreation \\
    	\textit{gr} & Function fit to Google mobility data for grocery and pharmacy \\
    	\textit{pa} & Function fit to Google mobility data for parks \\
    	\textit{tr} & Function fit to Google mobility data for transit stations \\
    	\hline
	\end{tabular}
\end{table}