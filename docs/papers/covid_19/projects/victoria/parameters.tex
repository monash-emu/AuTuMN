\section{Parameters}
\subsection{Non-age-stratified parameters}

\begin{longtable}[ht]{| >{\raggedright}p{4cm} | >{\raggedright}p{3cm} | p{6.8cm} |}
    \hline
    Parameter & Value & Rationale \\
    \endfirsthead
	\multicolumn{3}{c}{continuation of parameters table}\\
    \endhead
    \hline Incubation period & Calibration parameter, truncated normal distribution, mean 5.5 days & Estimates of the incubation period have included 5.1 days, 5.2 days and 4.8 days \cite{RN7, RN12, RN8, RN10}. A systematic review \cite{RN13} found that data are best fitted by a log-normal distribution (mean 5.8 days, CI 5.0 to 6.7, median 5.1 days). Our systematic review \cite{RN17} found that estimates of the mean incubation period have varied from 3.6 to 7.4 days. \\
    \hline
    Proportion of incubation period infectious & 50\% &     Infectiousness is considered to be present throughout a considerable proportion of the incubation period, based on analyses of confirmed source-secondary pairs \cite{RN23} and early findings that the incubation period was similar to the serial interval \cite{RN7}. The study of source-secondary pairs was also the primary reference cited by a review of the infectious period that identified studies that quantified the pre-symptomatic period, which concluded that the median pre-symptomatic period could range from less than one to four days \cite{RN14}. \\
    \hline
    Active period (regardless of detection/isolation, for clinical strata 1 to 3) &
    Calibration parameter, truncated normal distribution, mean 8 days &
    This quantity is difficult to estimate, given that identified cases are typically quarantined. Studies in settings of high case ascertainment and an effective public health response have suggested a duration of greater than 5.5 days \cite{RN10}. PCR positivity, which may continue for up to two to three weeks from the point of symptom onset \cite{RN23} \cite{RN14}, is difficult to interpret and does not necessarily indicate infectiousness. Consistent with these findings, the duration infectious for asymptomatic persons has been estimated at 6.5 to 9.5 days \cite{RN14} (although in our model, this would include the pre-symptomatic infectious period). \\
    \hline
    Proportion of infectious period before isolation or hospitalisation can occur & 
    0.333 &
    Assumed \\
    \hline
    Disease duration prior to admission for hospitalised patients not critically unwell (i.e. early active sojourn time, stratum 4) &
    7.7 days &
    Mean value from ISARIC cohort, as reported on 4\textsuperscript{th} October 2020 in Table 6 \cite{RN22}, and similar to the expected mean from earlier reports from ISARIC \cite{RN16}. This cohort represents high-income countries better than low and middle-income countries, with the United Kingdom contributing data on the greatest number of patients, followed by France. Earlier estimates of this quantity from China included 4.4 days \cite{RN7}. \\
    \hline
    Duration of hospitalisation if not critically unwell (late active sojourn time, stratum 4) &
    12.8 days &
    Mean value from the ISARIC cohort, as reported on 4\textsuperscript{th} October 2020 in Table 6 \cite{RN22}. \\
    \hline
    ICU duration (late active sojourn time, stratum 5) & 10.5 days &
    Mean duration of stay in ICU/HDU from ISARIC cohort for patients with complete data, as reported on 10\textsuperscript{th} October 2020 Table 6 \cite{RN22}. Many other studies reporting on the average duration of ICU stay suffer from right-truncation issues, often estimating 7-10 days length of stay. \\
    \hline    
    Duration of time prior to ICU for patients admitted to ICU & 
    10.5 days & 
    Calculated as the sum of the time from symptom onset to hospital admission (7.7 days above) plus the duration from hospital admission to ICU admission reported by October ISARIC report (2.8 days) \cite{RN22}. \\
    \hline
    Relative infectiousness of asymptomatic persons (per unit time with active disease) & 0.5 & Assumed \\
    \hline
    Relative infectiousness of persons admitted to hospital or ICU & 0.2 & Assumed \\
    \hline
    Relative infectiousness of identified persons in isolation & 0.2 & Assumed \\
    \hline
    Proportion of hospitalised patients ever admitted to ICU & 0.17 & Assumed \\
    \hline
	\caption{\textbf{Universal (non-age-stratified) model parameters.} Point estimates are used as model parameters except where ranges are indicated in calibration parameter table below in calibration table.}
	\title{Universal model parameters.}
	\label{tab:params}
\end{longtable}
