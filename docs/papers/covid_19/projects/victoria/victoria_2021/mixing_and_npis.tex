\section{Simulation of local NPI implementation during Victoria's second wave}

\subsection{School closures}
The effect of Victorian school closures is captured through the timeline presented in Table \ref{tab:school_timeline}.


\begin{table}[ht]
\renewcommand{\baselinestretch}{1}
	\begin{tabular}[ht]{| p{4.2cm} | p{6.2cm} | p{3.2cm} |}
	\hline
		Date of change & Policy change & Modification applied to school contacts contribution to mixing matrix, \(s(t)\) \\
		\hline
		From model start & Onsite learning & 1 \\
		\hline
		16\textsuperscript{th} July & Schools close for lockdown 5 & 0.1 \\
		\hline
		27\textsuperscript{th} July & Schools re-open after lockdown 5 & 1 \\
		\hline
		5\textsuperscript{th} August & Schools close for lockdown 6 & 0.1 \\
		\hline
		18\textsuperscript{th} September & Four of 13 year levels (foundation to year 2 and year 12) return in regional Vic & 0.30769 (regional only) \\
		\hline
    \end{tabular}
    \title{Timeline used to implement Victorian school closure policies.}
    \caption{\textbf{Timeline used to implement Victorian school closure policies.} The function is applied to both metropolitan and regional services.}
    \label{tab:school_timeline}
\end{table}

\subsection{Macrodistancing in workplaces and other locations}
The functions applied here are determined by the Google mobility data according to Table \ref{tab:mobility_map}, as described above, but are applied separately for each health service. Because Google mobility data pertain to local government areas (LGAs), whereas health service clusters may receive patients from across the state, it was necessary to map mobility data to services. Health service clusters' overall mobility values in each location were calculated using a weighted average of LGA mobility values according to the historical pattern of the origin of patients presenting to services within each service.

As a hypothetical example, if 50\% of patients historically presenting to Barwon South West health services come from the City of Geelong, the mobility data for the City of Geelong will contribute 50\% of the Google mobility estimate of Barwon South West.

Historical patterns of patient presentations by health service cluster were provided by the Victorian Department of Health.

\subsection{Microdistancing approach}
In this application to Victoria, the microdistancing function \(m(t)\) is comprised of two components: physical distancing and face coverings. Both physical distancing and face coverings micro-distancing are applied to the three non-household locations, such that the microdistancing function for non-household locations is given by: \[m(t)=d(t)^2\times f(t)^2\]
The two interventions are assumed to be independent and so are multiplicative. As for the macrodistancing functions, the two functions of time are squared to represent their effects on both the infector and the infectee in any potentially infectious interaction.
The face covering function in this analysis is set at the constant value of 0.84, because recent proportions of YouGov survey participants responding ``always" to the question ``Thinking about the last 7 days, have you worn a face mask outside your home (e.g. when on public transport, going to a supermarket, going to a main road)?" have been consistently around 84\%.
For the physical distancing function, the proportion of participants reporting zero contacts within two metres in Victoria has increased from around 16\% to 24\% over recent months. A hyperbolic tan function was used to fit these data, as presented in Figure \ref{fig:micro_physical}.

\begin{figure}[ht]
 	\resizebox{1\textwidth}{!}{\includegraphics[scale=1]{../covid_19/projects/victoria/victoria_2021/physical_distancing_fit.png}}
    \caption{\textbf{Physical distancing micro-distancing function with data used for fitting.}}
	\label{fig:micro_physical}
	\title{Physical distancing micro-distancing function for metropolitan Melbourne services with data used for fitting.}
\end{figure}
