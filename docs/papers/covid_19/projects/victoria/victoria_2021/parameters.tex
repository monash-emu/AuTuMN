\section{Parameters}
\subsection{Non-age-stratified parameters}

\begin{longtable}[ht]{| >{\raggedright}p{4cm} | >{\raggedright}p{3cm} | p{6.8cm} |}
    \hline
    Parameter & Value & Rationale \\
    \endfirsthead
	\multicolumn{3}{c}{continuation of parameters table}\\
    \endhead
    \hline Incubation period & Calibration parameter, truncated normal distribution, mean 6.1 days, standard deviation 0.8 days & Prior distribution taken from marginal posterior distribution from 2020 analysis \\
    \hline
    Proportion of incubation period infectious & 50\% & Infectiousness is considered to be present throughout a considerable proportion of the incubation period, based on analyses of confirmed source-secondary pairs \cite{RN23} and early findings that the incubation period was similar to the serial interval \cite{RN7}. The study of source-secondary pairs was also the primary reference cited by a review of the infectious period that identified studies that quantified the pre-symptomatic period, which concluded that the median pre-symptomatic period could range from less than one to four days \cite{RN14}. \\
    \hline
    Active period (regardless of detection/isolation, for clinical strata 1 to 3) &
    Calibration parameter, truncated normal distribution, mean 6.4 days, standard deviation 0.7 days &
    Prior distribution taken from marginal posterior distribution from 2020 analysis \\
    \hline
    Proportion of infectious period before isolation or hospitalisation can occur & 
    0.333 &
    Assumed \\
    \hline
    Disease duration prior to admission for hospitalised patients not critically unwell (i.e. early active sojourn time, stratum 4) &
    7.7 days &
    Mean value from ISARIC cohort, as reported on 4\textsuperscript{th} October 2020 in Table 6 \cite{RN22}, and similar to the expected mean from earlier reports from ISARIC \cite{RN16}. This cohort represents high-income countries better than low and middle-income countries, with the United Kingdom contributing data on the greatest number of patients, followed by France. Earlier estimates of this quantity from China included 4.4 days \cite{RN7}. \\
    \hline
    Duration of hospitalisation if not critically unwell (late active sojourn time, stratum 4) &
    7.8 days &
    FluCAN monthly Covid Epi Report to CDNA, 30\textsuperscript{th} August 2020 \\
    \hline
    ICU duration (late active sojourn time, stratum 5) & 5.9 days &
    SPRINT-SARI Australia Project \\
    \hline
    Duration of time prior to ICU for patients admitted to ICU & 
    10.5 days & 
    Calculated as the sum of the time from symptom onset to hospital admission (7.7 days above) plus the duration from hospital admission to ICU admission reported by October ISARIC report (2.8 days) \cite{RN22}. \\
    \hline
    Relative infectiousness of persons admitted to hospital or ICU & 0.2 & Assumed \\
    \hline
    Relative infectiousness of identified persons in isolation & 0.2 & Assumed \\
    \hline
    Clinical effectiveness of one vaccine dose & 0.49 & Mean of one dose effectiveness for BNT162b2 and ChAdOx1 reported in www.medrxiv.org/content/ 10.1101/2021.08.18.21262237v1 \\
    \hline
    Clinical effectiveness of full course of vaccination & 0.775 & Mean of full course effectiveness for BNT162b2 and ChAdOx1 reported in www.medrxiv.org/ content/10.1101/2021.08.18.21262237v1 \\
    \hline
    Proportion of effect of vaccination mediated through prevention of infection & 0.95 & Similar estimates for clinical effectiveness and effectiveness in preventing transmission \\
    \hline
    Reduction in infectiousness of one vaccine dose & 0.56 & Mean of one dose effectiveness for BNT162b2 and ChAdOx1 reported in www.medrxiv.org/content/ 10.1101/2021.08.18.21262237v1 \\
    \hline
    Reduction in infectiousness of full course of vaccination & 0.77 & Mean of full course effectiveness for BNT162b2 and ChAdOx1 reported in www.medrxiv.org/content/ 10.1101/2021.08.18.21262237v1 \\
    \hline    
	\caption{\textbf{Universal (non-age-stratified) model parameters.} Point estimates are used as model parameters except where ranges are indicated in calibration parameter table below in calibration table. Note that all vaccination-related parameters pertain specifically to Delta.}
	\title{Universal model parameters.}
	\label{tab:params}
\end{longtable}
