\section{Implementation of the roadmap}

The scenario analysis of implementation of the roadmap on schedule is implemented as described in Table \ref{tab:roadmap}.
\begin{table}[ht]
\renewcommand{\baselinestretch}{1}
	\begin{tabular}[ht]{| p{1.8cm} | p{3.6cm} | p{3.6cm} | p{3.6cm} |}
	\hline
		Phase & Date & Change & Implementation \\
		\hline
		A & \ordinalnum{5} October & VCE Units 3 and 4 onsite (Metropolitan Melbourne) & One thirteenth of 90\% remote education onsite \\
		\hline
		A & \ordinalnum{18} October & Prep back three days, years 1 and 2 back two days (Metropolitan Melbourne) & 2.4 thirteenth of 90\% remote education onsite \\
		\hline
		B & \ordinalnum{26} October & All students back at least part-time & 80\% onsite education \\
		\hline
		B & \ordinalnum{26} October & Social, recreational, personal care and religious changes & 10\% return to normal workplace mobility, 20\% return to normal other locations mobility \\
		\hline
		C & \ordinalnum{5} November & Social, recreational, work, personal care, retail, religious changes & 25\% return to normal workplace mobility, 40\% return to normal other locations mobility \\
		\hline
		D & \ordinalnum{19} November & Changes consistent with National Plan & 50\% return to normal workplace mobility, 60\% return to normal other locations mobility \\
		\hline		
    \end{tabular}
    \title{Timeline used to implement Victorian school closure policies.}
    \caption{\textbf{Timeline used to implement Victorian roadmap scenario.} Changes are applied to both Metropolitan and Regional areas except where indicated.}
    \label{tab:roadmap}
\end{table}