\section{Case detection and isolation} \label{cdr}

\subsection{Determining the proportion of cases detected}
We calculate a time-varying case detection rate, being the proportion of all symptomatic cases (clinical strata 2 to 5) that are detected (clinical strata 3 to 5). This proportion is informed by the number of tests performed using the following formula:

\[CDR(time)=1-e^{-shape \times tests(time)}\]

\textit{time} is the time in days from the 31\textsuperscript{st} December 2019 and \textit{tests(time)} is the number of tests per capita done on that date. To determine the value of the shape parameter, we solve this equation based on the assumption that a certain daily testing rate \textit{tests(t)} is associated with a certain \textit{CDR(t)}. Solving for \textit{shape} yields:

\[shape = \frac{-log(1 - CDR(t))}{tests(t)}\]

That is, if it is assumed that a certain daily per capita testing rate is associated with a certain proportion of symptomatic cases detected, we can determine \textit{shape}.
As this relationship is not well understood and unlikely to be consistent across all settings, we vary the \textit{CDR} that is associated with a certain per capita testing rate during uncertainty/calibration.
Given that the \textit{CDR} value can be varied widely, the purpose of this is to incorporate changes in the case detection rate that reflect the historical profile of changes in testing capacity over time.

The proportion of persons entering clinical stratum 3 is calculated once the \textit{CDR} is known, along with the proportion of all incident cases hospitalised (strata 4 and 5).

\subsection{Isolation of detected cases}
As described in the Section \ref{clin} above, as infected persons progress from the early to the late stage of active COVID-19, infectiousness is reduced for those in the detected strata (3 to 5) to reflect case isolation.
