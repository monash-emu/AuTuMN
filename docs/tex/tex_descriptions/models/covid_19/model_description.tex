\section{Base model construction}
\subsection{Platform for infectious disease dynamics simulation}

We developed a deterministic compartmental model of COVID-19 transmission using the AuTuMN platform,
publicly available https://github.com/monash-emu/AuTuMN/.
Our repository allows for the rapid and robust creation and stratification of models of infectious disease epidemiology and includes pluggable modules to simulate heterogeneous population mixing, demographic processes, multiple circulating pathogen strains, repeated stratification and other dynamics relevant to infectious disease transmission.
The platform was created to simulate TB dynamics, being an infectious disease whose epidemiology differs markedly by setting, such that considerable flexibility was desirable \cite{RN18}.
We have progressively developed the structures of our platform over recent years, and further adapted it to be sufficiently flexible to permit simulation of other infectious diseases, such as COVID-19. A similar model has been applied to several countries of the Asia-Pacific, with the application to the Philippines (without structure to represent geospatial stratification or contact tracing) previously described.\cite{RN79}

\subsection{Base COVID-19 model}
Using the base framework of an SEIR model (susceptible, exposed, infectious, removed), we split the exposed and infectious compartments into two sequential compartments each (SEEIIR). The two sequential exposed compartments represent the non-infectious and infectious phases of the incubation period, with the latter representing the ``presymptomatic" phase such that infectiousness occurs during three of the six sequential phases. For this reason, ``active" is a more accurate term for the two sequential ``I" compartments and is preferred henceforward. The two infectious compartments represent early and late phases of active disease, during which symptoms occur if the disease episode is symptomatic, and allow explicit representation of notification, case isolation, hospitalisation and admission to ICU. The ``active" compartment also includes some persons who remain asymptomatic throughout their disease episode, such that these compartments do not map directly to either persons who are infectious or those who are symptomatic (Figure \ref{fig:seeiir}).

\begin{figure}[ht]
    \includegraphics[width=\textwidth]{../covid_19/covid_19_seeiir.pdf}
   \title{Unstratified compartmental model structure.}
    \caption{\textbf{Unstratified compartmental model structure.} S = susceptible, E = exposed, I = active, R = recovered/removed. Depth of pink/red shading indicates the infectiousness of the compartment.}
    \label{fig:seeiir}
\end{figure}

The latently infected and infectious presymptomatic periods together comprise the incubation period, with the incubation period and the proportion of this period for which patients are infectious defined by input parameters described below. In general, two sequential compartments can be used to form a gamma-distributed profile of transition to infectiousness following exposure if the progression rates for these two compartments are equal, although in implementing this model the relative sojourn times in the two sequential compartments usually differed. Nevertheless, the profiles implemented are broadly consistent with the empirically observed log-normal distribution of individual incubation periods \cite{RN13}.

The transition from early active to late active represents the point at which patients are detected (for those persons for whom detection does eventually occur) and isolation then occurs from this point forward (i.e. applied during the late disease phase only). This transition point is also intended to represent the point of admission to hospital or transition from hospital ward to intensive care for patients for whom this occurs (see Section \ref{clin}).
