\section{Calculation of outputs}

\subsection{Incidence}
Incidence is calculated as any transitions into the early active compartment (\textit{``I"}).

\subsection{Hospital occupancy}
This is calculated as the sum of three quantities:
\begin{enumerate}
    \item All persons in the late active compartment in clinical stratum 4, representing those admitted to hospital but never critically unwell.
    \item All persons in the late active compartment in clinical stratum 5, representing those currently admitted to ICU.
    \item A proportion of the early active compartment in clinical stratum 5, representing those who will be admitted to ICU at a time in the future. This proportion is calculated as the quotient of \textit{1)} the difference between the pre-ICU period and the pre-hospital period for clinical stratum 4, divided by \textit{2)} the total pre-ICU period. That is, a proportion of the pre-ICU period is considered to represent patients in hospital who have not yet been admitted to ICU.
\end{enumerate}

\subsection{ICU occupancy}
This is calculated as all persons in the late active compartment in clinical stratum 5.

\subsection{Recovered proportion}
This is calculated as the proportion of the population in the recovered (\textit{``R"}) compartment. Although very similar numerically to the attack rate, persons who died of COVID-19 are not included in the denominator.

\subsection{COVID-19-related mortality}
This is calculated as all transitions representing death, exiting the model. This is implemented as depletion of the late active compartment.

\subsection{Notifications}
Local case notifications are calculated as transitions from the early to the late active compartment for clinical strata 3 to 5.
