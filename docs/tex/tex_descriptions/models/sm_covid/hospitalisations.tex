The risk of hospitalisation given infection is expected to vary dramatically by setting.
For example, different countries may have different criteria for whether or not a COVID-19 
patient should be admitted to a hospital. This makes it difficult to provide accurate 
estimates of hospitalisation rates for multiple countries. 

This is why we introduce a universal indicator named ``hospital pressure'' in our analysis. This indicator
is obtained by considering the age-specific risk of hospitalisation given infection observed in the first year
of the pandemic in the Netherlands, adjusted for vaccination status and for the infecting strain (\textcolor{red}{See Table XX}).
The ``hospital pressure'' indicator can therefore be interpreted as the level of hospital occupancy that
would be observed in the analysed country if the rates of hospitalisation given infection in this country were the same
as in the Netherlands. This indicator is expected to be roughly proportional to the actual hospital occupancy level of the 
studied country. Note that this indicator is used in order to make comparisons between scenarios such that one should interpret the relative
differences between scenarios rather than the absolute values of the indicator. 

Let us denote $i_{a,v,s}(t)$ the number of new disease episodes estimated to start at time $t$ for people aged $a$ with vaccination status $v$
and infected with strain $s$. The number of new hospital admissions occurring at time $t$ is calculated using the following
convolution product:
\begin{equation}
 \eta(t) = \sum_{a,v,s} \kappa_{a,v,s} \int_{u \geq 0}  i_{a,v,s}(t-u)g_{h}(u) du   \quad,
 \end{equation}
where $\kappa_{a,v,s}$ is the risk of hospitalisation given infection for age $a$, vaccination status $v$ and strain $s$ 
based on the Netherlands data, and $g_h$ is the probability density function of the statistical distribution chosen to represent the 
time from symptom onset to hospitalisation (\textcolor{red}{See Table XX}). 

We then compute the ``hospital pressure'' quantity $h$, which is an indicator of hospital occupancy level, by combining the number of new 
hospital admissions $\eta$ with the statistical distribution used to model hospital stay duration:
\begin{equation}
h(t) = \int_{u \geq 0}  \eta(t-u) (1 - \tau(u)) du   \quad,
\end{equation}
where $\tau$ is the cumulative density function of the statistical distribution chosen to represent the 
hospital stay duration (\textcolor{red}{See Table XX}). 
