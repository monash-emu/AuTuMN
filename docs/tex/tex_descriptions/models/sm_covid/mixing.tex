The model captures changes in social interactions over time through a dynamic age-specific mixing matrix. The following
sections describe how this matrix is defined and how it captures the different non-pharmaceutical interventions implemented
in the analysed countries.

\paragraph{Reference mixing matrices}
We extracted country-specific contact matrices using the \textit{socialmixr} R package (version 0.1.8 ) which derives social mixing matrices from 
contact survey data. These matrices provide the average numbers of contacts per day between different age groups, disaggregated by the following 
locations: home, school, work, other locations. We considered only the surveys conducted before the COVID-19 pandemic in order to obtain 
raw contact rate estimates that are not affected by COVID-19-related mobility changes. The next sections describe how we then adjusted 
the raw matrices to account for dynamic mobility.

For countries where contact survey data were not available, we used the matrices from a reference country among 
those with available data (\textcolor{red}{Still need to come up with a plan for this}). The reference matrices were 
then adjusted to account for age distribution differences between the modelled country and the reference country. For each 
location $L$ (home, school, work, other locations) the age-specific contact matrix of the modelled country $C^L = (c_{i,j}^L) \in R_{+}^{5 \times 5}$ 
is defined such that $c_{i,j}^L$ is the average number of contacts that a typical individual aged $i$ has with individuals aged $j$. 
Let us denote $Q^L = (q_{i,j}^L) \in R_{+}^{5 \times 5}$ the original matrix associated with location $L$ for the reference country, where $q_{i,j}^L$ is defined using the same 
convention as for $c_{i,j}^L$. Let $\pi_j$ denote the proportion of people aged $j$ in the modelled country in 2020, and $\rho_j$ the proportion of people aged $j$ in the reference country 
at the time of the contact survey. 
The contact matrices $C^L$ were obtained from:
$$
c_{i,j}^L = q_{i,j}^L \times \frac{\pi_j}{\rho_j} . 
$$

The overall contact matrix (before adjustments for mobility changes) results from the summation of the four location-specific 
contact matrices: \(C_{0}=C_{H}+C_{S}+C_{W}+C_{L}\), where \(C_{H}\), \(C_{S}\), \(C_{W}\) and \(C_{L}\) are the age-specific
contact matrices associated with households, schools, workplaces and other locations, respectively.

\paragraph[blabla]{Modifications of contact rates over time}
To capture mobility changes over time, the contributions of the matrices \(C_{S}\), \(C_{W}\) and \(C_{L}\) vary with time such that the input contact matrix can be written:
\[C(t)= C_{H}+ s(t)^{2}C_{S}+ w(t)^{2}C_{W}+l(t)^{2}C_{L}\]

The modifying functions $s$ (for schools), $w$ (for work) and $l$ (for other-locations) are each squared to capture the effect of the mobility changes on 
both the infector and the infectee in any given interaction that could potentially result in transmission. 

\textbf{School closure/re-opening }
Reduced attendance at schools is represented through the function $s$, which represents the proportion of all school students 
currently attending on-site teaching. If schools are fully closed, \(s(t)=0\) and \(C_{S}\) does not contribute to the overall 
mixing matrix \(C(t)\). 
\(s(t)\) is derived from UNESCO data on school closures since the start of the COVID-19 pandemic. 

\textcolor{red}{Detail the UNESCO data and the partial closure assumption. Also include an image here}




\textbf{Workplace closures}
Workplace closures are represented by quadratically reducing the contribution of workplace contacts to the total mixing matrix over time. This is achieved through the scaling term \(w(t)^{2}\) which modifies the contribution of \(C_{W}\) to the overall mixing matrix \(C(t)\). The profile of the function \(w(t)\) is set by fitting a polynomial spline function to Google mobility data for workplace attendance (Table \ref{tab:mobility_map}).

\textbf{Community-wide movement restriction}
Community-wide movement restriction (or ``lockdown") measures are represented by proportionally reducing the contribution of the other locations contacts to the total mixing matrix over time. This is achieved through the scaling term \(l(t)^{2}\) which modifies the contribution of \(C_{L}\) to the overall mixing matrix \(C(t)\). The profile of the function \(l(t)\) is set by fitting a polynomial spline function to an average of Google mobility data for various locations, as indicated in Table \ref{tab:mobility_map}.

\begin{table}[ht]
\renewcommand{\baselinestretch}{1}
    \begin{tabular}{| p{4.4cm} | p{4.4cm} | p{5cm} |}
        \hline
        \textbf{location} & \textbf{Approach} & \textbf{Google mobility types} \\
        \hline
        School & Policy response & Not applicable \\
      \hline
      Household & Constant & Not applicable \\
      \hline
      Workplace & Google mobility & Workplace \\
      \hline
      Other locations & Google mobility & 
      Unweighted average of: \begin{itemize}
			\item Retail and recreation
          \item Grocery and pharmacy
          \item Parks
          \item Transit stations
      \end{itemize}\\
      \hline
    \end{tabular}
    \title{Mapping of Google mobility data to contact locations.}
    \caption{\textbf{Mapping of Google mobility data to contact locations.}}
    \label{tab:mobility_map}
\end{table}

The contribution of household contacts to the overall mixing matrix is fixed over time. Although Google provides mobility 
estimates for residential contacts, the nature of these data are different from those for each of the other Google mobility 
types. They represent the time spent in that location, as opposed to other categories, which measure a change in total visitors 
rather than the duration. The daily frequency with which people attend their residence is likely to be close to one and we 
considered that household members likely have a daily opportunity for infection with each other household member regardless of
the background level of mobility. Therefore, we did not implement a function to scale the contribution of household contacts 
to the mixing matrix with time.

