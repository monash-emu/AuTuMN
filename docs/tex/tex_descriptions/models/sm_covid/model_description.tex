% This file describes the general structure of some of our sm_sir models.
% Note that it is possible to construct models in many different ways using the sm_sir code.
% Therefore, this description should be used with caution, because it only describes one possible model configuration.

\section{Model description}

\subsection{General approach}
We use a semi-mechanistic compartmental model of COVID-19 transmission governed by ordinary differential equations (ODEs). 
Our model captures important factors of COVID-19 transmission and disease such as age-specific characteristics, 
heterogeneous mixing, vaccination and the emergence of different variants of concern. 
The ODE-based model is used to capture only states relevant to transmission, whereas hospitalisations and deaths are
estimated through a convolution process. This process combines the model-estimated disease incidence with 
statistical distributions modelling the time to hospitalisation, the hospital stay duration and the time to death.
This approach presents two main advantages. First it reduces the complexity of the dynamic system relying on 
numerical solving of ODEs, which is computationally expensive. Second, the convolution approach allows for more flexibility 
and produces more realistic assumptions regarding the timings of hospitalisations and deaths, 
compared to what could be achieved with a simple compartmental approach. The following sections describe the model in details.

% ____________________________________________________
% Let's talk about the transmission model
%______________________________________________________

\subsection{Transmission model}
\label{trans} 
\subsubsection{Compartment types and sequence}
Model compartments represent sequential progressions through the processes of 
infection with, progression through, and recovery from the phases of SARS-CoV-2
infection and COVID-19 disease. The following types of compartments are implemented:
\begin{itemize}
    \item Susceptible
    \begin{itemize}
        \item Persons never previously infected with SARS-CoV-2 during or before the model simulation period
    \end{itemize}
    \item Latent
    \begin{itemize}
        \item Persons recently infected with SARS-CoV-2, but not in the active phase of the disease yet.
        \item These individuals may still be infectious (see details in next paragraph).
    \end{itemize}
    \item Active
    \begin{itemize}
        \item Persons with active COVID-19 who are currently infectious.
    \end{itemize}
    \item Recovered
    \begin{itemize}
        \item Persons recovered from COVID-19 during the model simulation period       
        \item Reinfection from these compartments is permitted through exposure to a different
        strain than the one that most recently infected the individual (see strain stratification section for details).
    \end{itemize}
\end{itemize}
The base model structure consists of a sequence of one susceptible compartment ($S$), four latent compartments ($E_1$, ..., $E_4$), four active disease compartments ($I^1$, ..., $I^4$) and one recovered compartment ($R$) (Figure \ref{fig:se4i4r}).

\begin{figure}[ht]
    \begin{center}
    \includegraphics[width=0.75\textwidth]{../../tex_descriptions/models/sm_covid/sm_covid_se4i4r.pdf}
    \end{center}
    \caption{Compartmental model structure. 
    S = Susceptible, E = Exposed / Latent, I = Active disease, R = Recovered.
    Stratification by age and vaccination status are not shown here.
    } 
    \label{fig:se4i4r}
\end{figure}

The main rationale for using four serial compartments for both the latent and active states is to achieve an Erlang distribution for the time spent in each of these states. This distribution is more realistic than
the exponential distribution that would have been associated with a single compartment, because the Erlang distribution does not have a large density mass around 0 and is not heavy-tailed. Figure [\textcolor{red}{add Fig Ref}] 
illustrates the modelled distributions of the 
The four active disease compartments have all exactly the same characteristics. However, the last two latent compartments ($E^3$ and $E^4$) are infectious whereas the first two ($E^1$ and $E^2$) are not. We further assume that 
the infectious latent compartments are half as infectious as the active disease compartments.

\subsubsection{Model stratification by age}
\label{age}
\section{Age stratification}
% Note that this will vary for every application, so will need to be edited - not sure of how best to manage this:
All compartments of the base compartmental structure were stratified by age into the following age bands:
\begin{itemize}
    \item 0 to 4 years
    \item 5 to 9 years
    \item 10 to 14 years
    \item 15 to 19 years
    \item 20 to 24 years
    \item 25 to 29 years
    \item 30 to 34 years
    \item 35 to 39 years
    \item 40 to 44 years
    \item 45 to 49 years
    \item 50 to 54 years
    \item 55 to 59 years
    \item 60 to 64 years
    \item 65 to 69 years
    \item 70 to 74 years
    \item 75 years and above
\end{itemize}
Demographic processes, including births, ageing and non-infection-related deaths 
are not simulated, given the timeframes considered in this simulation.


\subsubsection{Capturing the effects of vaccination}
\section{Vaccination history stratification}
% Referring to the immunity stratification as the vaccination stratification here, for the current application
History of vaccination is captured by stratifying all model compartments by vaccination status.
Two vaccination strata are included to represent those who have received at least two doses of a COVID-19 vaccine,
and those who have not.


\subsubsection{Modelling multiple viral strains}
\subsection{Modelling Variants of Concern (VoC)}
To consider the effects of VoC on infection dynamics and the vaccination programs, we explicitly simulated two competing strains to represent 1) the wild-type or ancestral virus, and 2) all VoC strains, where the VoC were assumed to be associated with increased transmissibility only, which is set to be calibrated from the model. Therefore, we do not differentiate between different variants and assume the single VoC strain that is modelled represents all currently circulating strains. Susceptible individuals can be infected with either the wild-type or VoC strain and infectious individuals contribute to the force of infection with their respective infecting strain only. VoC strains are seeded into the model such that one additional person per day is infected with the VoC strain for a duration of ten days, with the time that this ten-day period commences varied during model calibration. 


\subsubsection{Dynamic social mixing}
Mixing matrices.

Google data.

School closure.

\subsubsection{Random transmission adjustment}
The risk of SARS-CoV2 transmission per contact is adjusted by a time-variant random process, making
the model semi-mechanistic. This random process reflects the fact that all the variations observed in the transmission
risk cannot be explained solely by the factors that are included explicitly in the model such as vaccination, dynamic mobility or new variants' emergence.
We therefore allow for random perturbations to the risk of transmission over time, although these perturbations are highly auto-correlated
to avoid significant changes over a short period of time.

We use a random walk with Gaussian update $W(t)$  defined by:

\begin{equation}
    \begin{split}
    W(0) & = 0 \\
    W(t+1) & \sim \mathcal{N}(W(t), \epsilon) \quad ,
    \end{split}
\end{equation}

where $\mathcal{N}$ denotes the normal distribution and where the standard deviation $\epsilon$ is automatically calibrated by the MCMC.
The random process is updated on a fortnightly basis and is transformed using the exponential function before being applied to the risk of transmission per contact (see Equation \ref{foi}). 

\subsubsection{Ordinary differential equations}
\label{ODEs}
Let us first introduce some new notations. The different age groups are indicated by the subscript $a$ and $\mathcal{A}$ represents
the set of all modelled age groups (see Section \ref{age}). The vaccination status is represented by the subscript $v$ and $\mathcal{V}$ is the set of 
vaccination statuses (i.e. $\mathcal{V}=$ \{``0'', ``1''\} where ``0'' represents unvaccinated people and ``1'' represents vaccinated people). The subscript $s$ is used to represent the different viral strains and $\mathcal{S}$ is the set 
of all strains (i.e. $\mathcal{S}=$ \{``wild-type'', ``delta'', ``omicron''\}). The average incubation period duration associated with strain $s$ is denoted $q_s$ and
the average duration of active disease is denoted $w$. The relative susceptibility to infection of individuals aged $a$ with 
vaccination status $v$ is denoted $\sigma_{a,v}$. The term $b_{a,v,s}(t)$ is used to designate the imported individuals of age $a$ and 
with vaccination status $v$ that are infected with strain $s$ (infection seeding). Vaccination is characterised by the age-specific
and time-variant per-capita vaccination rate $w_a$. Finally, $\chi_{s,\sigma}$ represents the relative susceptibility to infection
with strain $\sigma$ for individuals whose most recent infection episode was with strain $s$. Using these new notations combined 
with those previously introduced, we can describe the model with the following set of ordinary differential equations:

\begin{equation}
    \label{ode}
    \begin{split}
\frac{dS_{a,v}}{dt} & = -\sum_{s \in \mathcal{S}} \lambda_{a,s}(t)\sigma_{a,v} S_{a,v} + \Phi_v\omega_a(t)S_{a,v=0}  \quad , \\
\frac{dE_{a,v,s}^{1}}{dt} & =\lambda_{a,s}(t)\sigma_{a,v} \Bigl(S_{a,v}  +  \sum_{\sigma \in \mathcal{S}} \chi_{\sigma, s} R_{a,v,\sigma} \Bigr) - \frac{4}{q_{s}}E_{a,v,s}^{1} + \Phi_v\omega_a(t)E_{a,v=0,s}^1 \quad , \\
\frac{dE_{a,v,s}^{k}}{dt} & = \frac{4}{q_{s}}E_{a,v,s}^{k-1} - \frac{4}{q_{s}}E_{a,v,s}^{k} + \Phi_v\omega_a(t)E_{a,v=0,s}^k \quad,  \forall k \in \{2,3,4\} , \\
\frac{dI_{a,v,s}^{1}}{dt} & = \frac{4}{q_{s}}E_{a,v,s}^{4} - \frac{4}{w}I_{a,v,s}^{1} + b_{a,v,s}(t) + \Phi_v\omega_a(t)I_{a,v=0,s}^1 \quad , \\
\frac{dI_{a,v,s}^{k}}{dt} & = \frac{4}{w}I_{a,v,s}^{k-1} - \frac{4}{w}I_{a,v,s}^{k} + \Phi_v\omega_a(t)I_{a,v=0,s}^k \quad, \forall k \in \{2,3,4\} , \\
\frac{dR_{a,v,s}}{dt} & = \frac{4}{w}I_{a,v,s}^{4} - \sum _{\sigma \in \mathcal{S}} \lambda_{a,\sigma}(t)\sigma_{a,v} \chi_{s, \sigma} R_{a,v, s} + \Phi_v\omega_a(t)R_{a,v=0,s} \quad, 
    \end{split}
\end{equation}

where $\lambda_{a,s}$ represents the force of infection of strain $s$ affecting individuals of age $a$. The quantity $\Phi_v$ is a binary variable
used to switch between plus and minus signs depending on the vaccination status. It is equal to $1$ when
$v=$``1'' and $-1$ when $v=$``0''. In other words, $\Phi_v = 2\mathbbm{1}_{v=``1"} - 1$. 

The force of infection is calculated as:
\begin{equation}
    \label{foi}
 \lambda_{a,s}(t) = \beta e^{W(t)} \rho_{s} \sum_{\alpha \in \mathcal{A}} \sum_{v \in \mathcal{V}} c_{a,\alpha}(t) \Bigl( 0.5 \sum_{k=3}^{4} E_{\alpha,v,s}^{k} + \sum_{k=1}^{4} I_{\alpha,v,s}^{k} \Bigr) \quad .
\end{equation}
In the previous equation, $\beta$ represents the unadjusted risk of transmission per contact and $\rho_{s}$ is the relative transmissibility of strain $s$. 



% ____________________________________________________
% Now, let's talk about the convolution processes
%______________________________________________________
\subsection{Estimation of COVID-19-related hospital pressure and deaths}
The transmission model described in Section \ref{trans} provides estimates of COVID-19 incidence over time, disaggregated by age, vaccination status and strain. 
We combine these incidence estimates with the age-, vaccination- and strain-specific risks of hospitalisation and deaths as well as 
statistical distributions of time to events to compute COVID-19-related hospital pressure and deaths over time.

\subsubsection{COVID-19-related hospital pressure}
\label{hosp}
The risk of hospitalisation given infection was expected to vary markedly by setting.
For example, different countries may have different criteria for whether or not a COVID-19 
patient should be admitted to a hospital. This makes it difficult to provide accurate 
estimates of hospitalisation rates for multiple countries. 

For this reason we introduced a universal indicator named ``hospital pressure'' in our analysis. This indicator
was obtained by considering the age-specific risk of hospitalisation given infection observed in the first year
of the pandemic in the Netherlands, adjusted for vaccination status and for the infecting strain (Tables \ref{param_table} and \ref{agespec_table}).
The ``hospital pressure'' indicator can therefore be interpreted as the level of hospital occupancy that
would be observed in the analysed country if the rates of hospitalisation given infection in this country were the same
as for the Netherlands. This quantity is expected to vary proportionately with occupancy over time, providing an indicator of 
hospital pressure. Note that this indicator was used in order to make comparisons between scenarios, such that one should interpret the relative
differences between scenarios rather than the absolute values of the indicator. 

Let us denote $i_{a,v,s}(t)$ the number of new disease episodes estimated to start at time $t$ for people aged $a$ with vaccination status $v$
and infected with strain $s$. The number of new hospital admissions occurring at time $t$ was calculated using the following
convolution product:
\begin{equation}
 \eta(t) = \sum_{a,v,s} \kappa_{a,v,s} \int_{u \geq 0}  i_{a,v,s}(t-u)g_{h}(u) du   \quad,
 \end{equation}
where $\kappa_{a,v,s}$ is the risk of hospitalisation given infection for age $a$, vaccination status $v$ and strain $s$ 
based on the Netherlands data, and $g_h$ is the probability density function of the statistical distribution chosen to represent the 
time from symptom onset to hospitalisation (Table \ref{param_table}). 

We then computed the ``hospital pressure'' quantity $h$, which is an indicator of hospital occupancy level, by combining the number of new 
hospital admissions $\eta$ with the statistical distribution used to model hospital stay duration:
\begin{equation}
h(t) = \int_{u \geq 0}  \eta(t-u) (1 - \tau(u)) du   \quad,
\end{equation}
where $\tau$ is the cumulative density function of the statistical distribution chosen to represent the 
hospital stay duration (Table \ref{param_table}). 
 

\subsubsection{COVID-19 deaths}
We estimate the number of COVID-19 deaths over time using a similar approach 
as for the hospital pressure indicator. We use the age-specific infection fatality rates reported in
ODriscoll et al. (\textcolor{red}{ADD REF HERE}), adjusted for vaccination status and for the infecting strain
to estimate COVID-19 mortality. Using the same notations as in Section \ref{hosp}, the number of COVID-19
deaths observed at time $t$ is obtained by:
\begin{equation}
\mu(t) = \sum_{a,v,s} ifr^{a,v,s} \int_{u \geq 0}  i^{a,v,s}(t-u)g_{d}(u) du   \quad,
\end{equation}
where $ifr^{a,v,s}$ is the risk of death given infection for age $a$, vaccination status $v$ and strain $s$, 
and $g_d$ is the probability density function of the statistical distribution chosen to represent the 
time from symptom onset to death (\textcolor{red}{See Table XX}). 

