Let us first introduce some new notations. The different age groups are indicated by the subscript $a$ and $\mathcal{A}$ represents
the set of all modelled age groups (see Section \ref{age}). The vaccination status is represented by the subscript $v$ and $\mathcal{V}$ is the set of 
vaccination statuses (i.e. $\mathcal{V}=$ \{``unvaccinated'', ``vaccinated''\}). The subscript $s$ is used to represent the different viral strains and $\mathcal{S}$ is the set 
of all strains (i.e. $\mathcal{S}=$ \{``wild-type'', ``delta'', ``omicron''\}). The average incubation period duration associated with strain $s$ is denoted $q_s$ and
the average duration of active disease is denoted $w$. The relative susceptibility to infection of individuals aged $a$ with 
vaccination status $v$ is denoted $\sigma_{a,v}$. The term $b_{a,v,s}(t)$ is used to designate the imported individuals of age $a$ and 
with vaccination status $v$ that are infected with strain $s$ (infection seeding). Finally, $\chi_{s,\sigma}$ represents the relative susceptibility to infection
with strain $\sigma$ for individuals whose most recent infection episode was with strain $s$. Using these new notations combined 
with those previously introduced, we can describe the model with the following set of ordinary differential equations:

\begin{equation}
    \label{ode}
    \begin{split}
\frac{dS_{a,v}}{dt} & = -\sum_{s \in \mathcal{S}} \lambda_{a,s}(t)\sigma_{a,v} S_{a,v} \quad , \\
\frac{dE_{a,v,s}^{1}}{dt} & =\lambda_{a,s}(t)\sigma_{a,v} \Bigl(S_{a,v}  +  \sum_{\sigma \in \mathcal{S}} \chi_{\sigma, s} R_{a,v,\sigma} \Bigr) - \frac{4}{q_{s}}E_{a,v,s}^{1} \quad , \\
\frac{dE_{a,v,s}^{k}}{dt} & = \frac{4}{q_{s}}E_{a,v,s}^{k-1} - \frac{4}{q_{s}}E_{a,v,s}^{k} \quad,  \forall k \in \{2,3,4\} , \\
\frac{dI_{a,v,s}^{1}}{dt} & = \frac{4}{q_{s}}E_{a,v,s}^{4} - \frac{4}{w}I_{a,v,s}^{1} + b_{a,v,s}(t) \quad , \\
\frac{dI_{a,v,s}^{k}}{dt} & = \frac{4}{w}I_{a,v,s}^{k-1} - \frac{4}{w}I_{a,v,s}^{k} \quad, \forall k \in \{2,3,4\} , \\
\frac{dR_{a,v,s}}{dt} & = \frac{4}{w}I_{a,v,s}^{4} - \sum _{\sigma \in \mathcal{S}} \lambda_{a,\sigma}(t)\sigma_{a,v} \chi_{s, \sigma} R_{a,v, s} \quad, 
    \end{split}
\end{equation}

where $\lambda_{a,s}$ represents the force of infection of strain $s$ affecting individuals of age $a$. This force of infection
is calculated as:
\begin{equation}
    \label{foi}
 \lambda_{a,s}(t) = \beta e^{W(t)} \rho_{s} \sum_{\alpha \in \mathcal{A}} \sum_{v \in \mathcal{V}} c_{a,\alpha}(t) \Bigl( 0.5 \sum_{k=3}^{4} E_{\alpha,v,s}^{k} + \sum_{k=1}^{4} I_{\alpha,v,s}^{k} \Bigr) \quad .
\end{equation}
In the previous equation, $\beta$ represents the unadjusted risk of transmission per contact and $\rho_{s}$ is the relative transmissibility of strain $s$.