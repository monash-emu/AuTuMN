The risk of SARS-CoV-2 transmission per contact was adjusted by a time-variant random process, making
the model semi-mechanistic. This random process reflects the fact that all the variations observed in the transmission
risk in the real world cannot be explained solely by the factors that are explicitly captured through our model inputs (such as vaccination, dynamic mobility or new variants' emergence).
We therefore allowed for random perturbations to the risk of transmission over time, although the random process was highly auto-correlated
to avoid unrealistic changes over a short period of time.

We used a random walk with Gaussian update defined by:

\begin{equation}
    \label{eq:random_process}
    \begin{split}
    W(0) & = 0 \\
    W(t+1) & \sim \mathcal{N}(W(t), 0.5) \quad ,
    \end{split}
\end{equation}

where $\mathcal{N}$ denotes the normal distribution.
The random process $W$ was updated every two months and was transformed using the exponential function before being applied to the risk of transmission per contact (see Equation \ref{foi}).
Finally, the contribution of the random process to the risk of transmission was squared in order to capture its effect
on both the susceptible and the infectious individuals (see Equation \ref{foi}). 