
We used the \textit{summer2} Python package (v 1.2.5) to implement the model. This is a domain specific library for 
compartmental epidemiological models that addresses a few of the key concerns as follows.

\subsubsection{Application Programming Interface (API)}
The model specification was done via a simple yet expressive Python API, 
while the numerical implementation was largely autogenerated by the \textit{summer2} package at runtime. 
This specification is composable via stratification classes and other reusable components, 
thus the complexity of the software is kept to a minimum, reducing cognitive overhead and greatly reducing the possibility for error.

\subsubsection{Optimising compiler}
The \textit{summer2} package uses the jax library as its computational backend, meaning that while the specification of models is done largely in Python, 
the model execution itself is transformed via an optimising compiler into fast native code.
This brings the model runtime from several seconds (for a naive implementation) to under 50ms per iteration, which was necessary to perform the computationally-intensive
calibration tasks described in Section \ref{calibration}.




