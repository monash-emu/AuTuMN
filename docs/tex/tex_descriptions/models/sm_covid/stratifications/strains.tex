
The model is stratified by ``strain'' to simulate the emergence of multiple variants of concern (VoC).
This approach explicitly represents multiple competing strains, each with a separate force of infection calculation.
We assume that VoCs can have different levels of transmissibility, incubation period's duration and disease severity 
(hospitalisation and death risks) compared to the ancestral COVID-19 strain. In addition, VoCs are assumed to escape 
immunity partially for both vaccination- and infection-related immunity. 

We assume that individuals previously infected with the wild-type strain can only be reinfected with the delta or 
omicron strains. However, such individuals have a reduced risk of infection with these variants compared to 
infection-naive individuals (68\% and 45\% reduction for delta and omicron, respectively) \textcolor{red}{[REF DOI: 10.1503/cmaj.211248]}.
We assume that individuals previously infected with the delta variant can only be reinfected with the omicron variant, 
with an infection risk reduced by 45\% compared to infection-naive individuals \textcolor{red}{[REF DOI: 10.1503/cmaj.211248]}. 
The other parameters used to represent strain-specific characteristics are presented in Table \ref{param_table}.

Seeding of each new strain into the model is achieved through the importation of a small number (10 per million population) of new infectious persons with the relevant strain into the model.
The seeding process is done over a ten-day period and the start of this period is set to the emergence date reported by GISAID.
