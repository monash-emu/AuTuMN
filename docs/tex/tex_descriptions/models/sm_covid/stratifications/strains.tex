
The model is stratified by ``strain'' to simulate the emergence of multiple variants of concern (VoC).
This approach explicitly represents multiple competing strains, each with a separate force of infection calculation.
We assume that VoCs can have different levels of transmissibility, incubation period's duration and disease severity 
(hospitalisation and death risks) compared to the ancestral COVID-19 strain. In addition, VoCs are assumed to escape 
immunity partially for both vaccination- and infection-related immunity. The parameters used to represent strain-specific
characteristics are presented in Table \ref{param_table} and \textcolor{red}{Table XXX for cross-immunity}.

Seeding of each new strain into the model is achieved through the importation of a small number (10 per million population) of new infectious persons with the relevant strain into the model.
The seeding process is done over a ten-day period and the start of this period is set between 30 days before and 30 days after the emergence date
as reported by GISAID \textcolor{red}{Table XXX}. The exact start date is automatically calibrated by the MCMC algorithm.
