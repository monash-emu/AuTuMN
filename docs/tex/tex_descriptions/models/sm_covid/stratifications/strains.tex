
The model was stratified by ``strain'' to simulate the emergence of multiple variants of concern (VoC).
This approach explicitly represents multiple competing strains, each with an independent force of infection calculation.
We assumed that VoCs can have different levels of transmissibility, incubation period and disease severity 
(hospitalisation and death risks) compared to the ancestral COVID-19 strain. In addition, VoCs were assumed to escape 
immunity partially for both vaccination- and infection-related immunity. 

We assumed that individuals previously infected with the wild-type strain could only be reinfected with the delta or 
omicron strains. However, such individuals have a reduced risk of infection with these variants compared to 
infection-naive individuals (68\% and 45\% reduction for delta and omicron, respectively) \textcolor{red}{[REF DOI: 10.1503/cmaj.211248]}.
We assumed that individuals previously infected with the delta variant could only be reinfected with the omicron variant, 
with an infection risk reduced by 45\% compared to infection-naive individuals \textcolor{red}{[REF DOI: 10.1503/cmaj.211248]}. 
The other parameters used to represent strain-specific characteristics are presented in Table \ref{param_table}.

Seeding of each new strain into the model was achieved through the importation of a small number (10 per million population) of new infectious persons with the relevant strain into the model.
The seeding process was done over a ten-day period and the start of this period was set to the emergence date reported by GISAID.
