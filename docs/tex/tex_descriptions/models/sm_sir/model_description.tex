% This file describes the general structure of some of our sm_sir models.
% Note that it is possible to construct models in many different ways using the sm_sir code.
% Therefore, this description should be used with caution, because it only describes one possible model configuration.

\section{Model Structure}

\subsection{General features}

Our COVID-19 model is a series of compartments representing transitions between states relevant to infection with SARS-CoV-2 and onward transmission of this virus.
Unlike previous iterations of this model (including our past publications on the epidemics in the Philippines \cite{caldwell-2021-a}, Malaysia \cite{jayasundara-2021} and Victoria, Australia \cite{trauer-2021-a}), our current model considers only states relevant to transmission.
Hospitalisation, admission to ICU and death are no longer represented as explicit model states, but are now calculated from model outputs through a convolution process.
The rationale for this approach is that the explicitly modelled states are reserved for the representation of processes relevant to epidemic transmission dynamics only.
Any other relevant outcomes are then calculated from the quantities that are tracked during the process of numerically solving the dynamic system (``derived outputs''), including through convolutions.
It should be noted that the process of calculating derived outputs is done after each model iteration, such that these quantities can still be compared to empirically observed outcomes and so used for calibration.

\subsection{Compartments}
Model compartments represent sequential progressions through the processes of infection with, progression through and recovery from the phases of SARS-CoV-2.
Reinfection is permitted in our model code structure, which is represented as transition from the recovered compartments back to the first infected compartment.
The following compartments are implemented:
\begin{itemize}
    \item Susceptible
    \begin{itemize}
        \item Persons never previously infected with SARS-CoV-2 during the model simulation period
    \end{itemize}
    \item Latent
    \begin{itemize}
        \item Persons recently infected with SARS-CoV-2, but not yet infectious
        \item This phase is divided into two sequential compartments
    \end{itemize}
    \item Infectious
    \begin{itemize}
        \item Persons with active COVID-19 who are potentially infectious to others
        \item This phase is divided into two sequential compartments
        \item The second of these two sequential compartments includes any persons who are identified through the health system and asked to isolate
        \item The infectiousness of the second compartment will be reduced to capture the effect of case isolation
    \end{itemize}
    \item Recovered
    \begin{itemize}
        \item Persons recovered from COVID-19 during the model simulation period
        \item This phase is divided into two sequential compartments
        \item Reinfection from these compartments is permitted and occurs at a different rate for the two sequential phases (if the reinfection rate is greater than zero)
        \item This compartment retains the stratification by strain (or variant of concern (``VoC'')), with the strain of the last infection episode used for classification
    \end{itemize}
\end{itemize}
