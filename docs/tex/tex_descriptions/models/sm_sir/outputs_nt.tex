\section{Outputs}

\subsection{Incidence}
Incidence represents the rate of onset of new disease episodes,
(regardless of whether symptoms or case detection occur).
It is calculated as the rate of transition between the early and late active model compartments.
Therefore, this indicator is slightly lagged relative to the onset of infectiousness,
and occurs at the same point in the disease episode as for case isolation
(for those persons for whom self-isolation occurs).
Thus, this indicator can be reasonably considered as relating to the point in a disease episode
at which significant symptoms would develop.
However, we stress that this indicator includes all disease episodes,
including asymptomatic and symptomatic undetected episodes.

\subsection{Notifications}
Notifications are derived from the calculation of incidence described above,
but limited to the detected clinical stratum (see \ref{clin}).
Therefore, this considers case detection as being interchangeable with notification
(with cases detected by the surveillance system being universally included
in the case notification data provided to us for model calibration).
A gamma distribution convolution is then applied to the detected case incidence to allow for the time lag
between symptom onset and case reporting.  % Refer to the parameters table for the specific parameters
The mass of the gamma distribution between each sequential pair of time points is
used in the calculation.

\subsection{COVID-19-related mortality}
% This has been adapted to the NT application to remove reference to different strain severity
COVID-19-related deaths are derived from the calculation of incidence described above,
but limited to the two symptomatic clinical strata (both detected and undetected).
Calculations are made separately for each age group and each vaccine-induced immunity stratum, 
given that these factors influence disease-induced mortality rates.
A convolution is applied to each age and immunity-specific 
symptomatic incidence rate calculation.
The age-specific base case fatality rate (CFR) 
is first determined as a model input,  % Refer to the parameters table
which is then adjusted according to the protection afforded by vaccine-induced immunity.
This is then be adjusted to better match the local epidemic characteristics through an
``adjuster'' parameter.
This parameter applies equally to each age group 
and acts to inflate or deflate the case fatality rate
adjusted for vaccine-induced immunity.
However, rather than directly multiplying the case fatality rate by this adjuster,
we apply the adjuster to the odds of death equivalent to the risk of death parameter.
That is, the adjusted case fatality rate is calculated as:
\[CFR_{adj} = \frac{CFR_{base} \times adjuster}{CFR_{base} \times (adjuster - 1) + 1}\]
% Next, the adjusted CFR is again adjusted for the severity of the strain
% responsible for the disease episode.
Having determined the age and immunity-specific CFR,
a gamma-distributed convolution is calculated such that the total density
of the distribution is equal to the CFR determined as described above.  % Refer to the parameters table 
As for notifications, the mass of the gamma distribution 
between each sequential pair of time points is
used in this calculation.

\subsection{Hospital admissions}
The rate of new hospital admissions is calculated from symptomatic incidence in 
an analogous way to that described for the mortality in the previous section.
Age and immunity-specific risks of hospitalisation are calculated,
with adjustment of all rates for local context as required, and a convolution is applied.
As for notifications and deaths, the mass of the gamma distribution 
between each sequential pair of time points is
used in the calculation.

\subsection{Hospital occupancy}
Having calculated the rate of hospital admissions,
a second gamma-distributed convolution is applied to this output to calculate
the total hospital occupancy from the admissions at every previous time point modelled.
Differing from the convolution used to calculate the rates of notifications,
the gamma distribution is used to determine the probability that an admitted patient
would remain in hospital after the time lag from the hospital admission rate being considered.
That is, the complement of the cumulative distribution function for the rate of hospital
discharge is applied.

\subsection{ICU admissions and occupancy}
The calcuation of ICU admissions and occupancy is analogous 
to that of hospital admissions and occupancy.
