\section{Vaccination history stratification}
% Referring to the immunity stratification as the vaccination stratification here, for the Bhutan application
History of vaccination is captured by stratifying all model compartments by vaccination status.
Two vaccination strata are included to represent those who have received at least two doses of a COVID-19 vaccine,
and those who have not.
The following are not considered in this approach to simulating vaccination:
\begin{itemize}
    \item Any effect of receiving a single dose of vaccine
    \item Waning of vaccine-induced immunity
    \item Any additional effect from receiving additional vaccine doses following the second dose
\end{itemize}
The effect of vaccination on transmission is to partially reduce the rate of infection for all persons at-risk of infection in the vaccinated stratum.
This includes both fully susceptible (never previously infected) persons,
as well as recovered persons who are at risk of reinfection.
Emerging variants of concern (VoCs) may escape this immunity, as described further below.
% To illustrate this further, the next steps would be to display the population distribution by stratum,
% and/or to explain the implementation of the dynamic proportions code

\section{Immunity-strain interaction}
The strains/VoCs implemented in the model may escape both vaccine-induced and natural immunity, as follows.

\subsection{Application to first infection}
The first infection episode with SARS-CoV-2 is represented through the transition from the susceptible
to the first latent compartment.
The susceptible compartment is not stratified by strain,
because these persons have no infection history.
Therefore, for infection of persons never previously infected,
only the vaccination-induced immunity status is relevant.
The rate of infection for a particular vaccination status and infecting strain is adjusted by the factor:
\[1 - p_{i} \times s_{j}\]
Where \(p_{i}\) represents the protection afforded by vaccination stratum \(_{i}\),
and \(s_{j}\) represents the degree of immune escape against vaccine-induced immunity for strain \(_{j}\).
Note that this assumes that the extent of immune escape for a given strain is the same
for all vaccination strata. 

\subsection{Application to re-infection}
Persons who have recovered from previous infection episodes remain in a compartment that is stratified
according to the last infecting strain.
The recovered compartment is replicated into early and late stages,
which can afford different levels of protection against re-infection.
The extent of protection must be explicitly specified for each strain-stage combination.
The rate of reinfection is adjusted by the factor:
\[(1 - p_{i} \times s_{j})\times (1 - n_{i,k,l})\]
Where the additional term \(n_{k,l,m}\) represents the cross protection
afforded by past infection with strain \(_{k}\)
against infecting strain \(_{l}\)
in vaccination stratum \(_{i}\).

\subsection{Adjusting age-specific severity parameters for immunity}
We use inputs for the age-specific case fatality rate and the age-specific hospitalisation rate from Nyberg et al., 
which provides estimates of these quantities in a population that already had substantial vaccine-induced immunity.
We therefore adjust these quantities for pre-existing immunity, to account for protection against death and hospitalisation.
We apply the following method to ensure that after adjusting the parameters for a specific age bracket for immunity,
the average parameter value would be equal to that reported by Nyberg et al. if weights were applied according to the
distribution of immunity in the population that was described in this study.
To do this, we estimate the proportion of the population in each of the three modelled immunity categories
in the United Kingdom around the mid-point of the study,
and the protection afforded by each of the immunity classes (see Table \ref{tab:immunity_weighting}).

\begin{table}
    \begin{threeparttable}
    \begin{tabularx}{\textwidth}{| X | X | X |}
        \hline
        \textbf{Immunity category} & \textbf{Proportion of population} \tnote{a} & \textbf{Estimated protection} \tnote{b} \\
        \hline
        No immunity, unvaccinated & 0.312 & 0 \\
        \hline
        Low immunity, vaccinated with two doses & 0.302 & 0.5 \\
        \hline
        High immunity, vaccinated with three doses & 0.386 & 0.85 \\
        \hline
	\end{tabularx}
	\caption{Values used to re-weight the severity estimates provided in Nyberg et al. to separate immunity classes.}
	\label{tab:immunity_weighting}
    \begin{tablenotes}
        \item[a] Estimated population distribution as at 16th December 2021 (study mid-point),
        taken from Our World In Data.
        \item[b] Estimated protection against hospitalisation or death following Omicron infection.
    \end{tablenotes}
    \end{threeparttable}
\end{table}

