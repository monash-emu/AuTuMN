\section{Vaccination history stratification}
\subsection{General approach}
% Referring to the immunity stratification as the vaccination stratification here, for the NT application
History of vaccination is captured by stratifying all model compartments by vaccination status.
Three vaccination strata are included to represent the following categories:
\begin{itemize}
    \item Persons who have received fewer than two doses of a COVID-19 vaccine (no vaccination or one dose only)
    \item Persons who have received exactly two doses of a COVID-19 vaccine
    \item Persons who have received at least three doses of a COVID-19 vaccine
\end{itemize}
Throughout the analysis period, all the vaccination schedules available in Australia
consisted of a two dose primary course.
The following are not considered in this approach to simulating vaccination:
\begin{itemize}
    \item Any effect of receiving a single dose of vaccine
    \item Waning of vaccine-induced immunity
    \item Any additional effect from receiving additional vaccine doses following the third dose
\end{itemize}
The effect of vaccination on transmission is to partially reduce 
the force of infection for all persons at-risk of infection in the vaccinated stratum.
This applies to both fully susceptible (never previously infected) persons,
as well as recovered persons who are at risk of reinfection.
Emerging variants of concern (VoCs) may escape this immunity, as described in the section on strains.

\subsection{Application to first infection}
The first infection episode with SARS-CoV-2 
is represented through the transition from the susceptible
to the first latent compartment.
The susceptible compartment is not stratified by strain,
because these persons have no infection history.
Therefore, for infection of persons never previously infected,
only the vaccination-induced immunity status is relevant.
The rate of infection for a particular vaccination status is adjusted by the factor:
\[1 - p_{i}\]
Where \(p_{i}\) represents the protection afforded by vaccination stratum \(_{i}\).

\subsection{Application to re-infection}
Persons who have recovered from previous infection episodes 
remain in a compartment that is stratified
according to the last infecting strain.
The recovered compartment is replicated into early and late stages,
which can afford different levels of protection against re-infection.
The rate of reinfection is adjusted by the factor:
\[(1 - p_{i})\times (1 - n_{k,l})\]
Where the additional term \(n_{k,l}\) represents the cross-protection
afforded by past infection with strain \(_{k}\)
against infecting strain \(_{l}\).
In this way, cross-protection can be thought of as the complement
of ``immune escape''/

\subsection{Adjusting age-specific severity parameters for immunity}
We use inputs for the age-specific case fatality rate and the age-specific hospitalisation rate from Nyberg et al., 
which provides estimates of these quantities in a population that already had substantial vaccine-induced immunity.
We therefore adjust these quantities for pre-existing immunity, to account for protection against death and hospitalisation.
We apply the following method to ensure that after adjusting the parameters for a specific age bracket for immunity,
the average parameter value would be equal to that reported by Nyberg et al. if weights were applied according to the
distribution of immunity in the population that was described in this study.
To do this, we estimate the proportion of the population in each of the three modelled immunity categories
in the United Kingdom around the mid-point of the study,
and the protection afforded by each of the immunity classes (see Table \ref{tab:immunity_weighting}).


\begin{table}
    \begin{threeparttable}
    \begin{tabularx}{\textwidth}{| X | X | X | X |}
        \hline
        \textbf{Immunity category} & \textbf{Proportion of population} \tnote{a} & 
        \textbf{Protection against hospitalisation} \tnote{b} & 
        \textbf{Protection against death} \tnote{b} \\
        \hline
        No immunity, unvaccinated & 0.312 & 0 & 0 \\
        \hline
        Low immunity, vaccinated with two doses & 0.302 & 0.5 & 0.8 \\
        \hline
        High immunity, vaccinated with three doses & 0.386 & 0.85 & 0.9 \\
        \hline
	\end{tabularx}
	\caption{Values used to re-weight the severity estimates provided in Nyberg et al. to separate immunity classes.}
	\label{tab:immunity_weighting}
    \begin{tablenotes}
        \item[a] Estimated population distribution as at 16th December 2021 (study mid-point),
        taken from Our World In Data.
        \item[b] Estimated protection against hospitalisation or death following Omicron infection.
    \end{tablenotes}
    \end{threeparttable}
\end{table}
