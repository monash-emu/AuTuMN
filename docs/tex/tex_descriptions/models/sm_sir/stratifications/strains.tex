\section{Strain stratification} \label{strain}
The model is stratified by ``strain'' or Variant of Concern (VoC),
adopting the summer approach to strain stratification through the \code{StrainStratification} object.
This general approach is described in full in the summer model documentation for to represent explicitly multiple competing strains,
each with a separate force of infection calculation (see \code{ModelRunner} object code where force of infection calculations are made and
\code{Stratification} object code where strain stratification can be requested).
This framework allows for independent calculations of the force of infection for each of the modelled strains,
with the compartmental infectiousness multipliers applied identically for all of the competing strains.
Additional strain features (such as differential infectiousness and duration of latency and active disease)
can then be implemented with the standard features of stratification,
such as adjustments to flow rates.

\subsection{Seeding of new VoC}
Seeding of each new strain into the model is achieved through the addition of an importation flow
that introduces a small number of new infectious persons with the new VoC into the model.
To achieve this, a new entry flow is added to the model with destination being the early infectious compartment.
The rate of this flow is a step function that steps from zero to the requested daily rate of seeding for the requested number of days,
before stepping back down to zero for the remainder of the period simulated.