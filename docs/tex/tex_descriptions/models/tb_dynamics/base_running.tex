% This file is the basic running of the model, just the population growth of Kiribati


\section{Population growth}
\subsection{Crude birth rate and death rate}
Births are modelled using time-variant crude birth rates that are multiplied by the modelled population 
size to determine the number of newborn individuals entering the model at each time. A time-variant 
and age-specific rate or non-TB-related mortality applies to all model compartments to simulate 
deaths from other causes than TB. We use estimates from the UN population division \footnote{https://population.un.org/wpp/Download/Standard/Mortality/}  to inform the 
birth and mortality rates. No data specific to the Marshall Islands were available, so we used the crude 
birth rates and mortality rates by age of the Federated States of Micronesia.
We also apply additional death rates to the compartments I and T to reflect mortality induced by TB 
disease
\begin{figure}[!htb]
    \includegraphics[width=\textwidth,height=\textheight,keepaspectratio]{images/cbr.png}
    \caption{The crude birth rate of Kiribati from 1950 to 2020.}
    \label{fig:cbr}
\end{figure}

\begin{figure}[!htb]
    \includegraphics[width=\textwidth,height=\textheight,keepaspectratio]{images/cdr.png}
    \caption{The death rate of Kiribati from 1950 to 2020, stratified by age group.}
    \label{fig:cdr}
\end{figure}

\subsection{Comparing modeled population with actual population}
\begin{figure}[!htb]
    \includegraphics[width=\textwidth,height=\textheight,keepaspectratio]{images/modelled_total.png}
    \caption{Comparing modelled population with actual population of Kirbati from 1950 to 2020. The red dots represent the actual population size of Kiribati,
     while the blue line represents the modelled population size}
    \label{fig:modelled_total}
\end{figure}

\begin{figure}[!htb]
    \includegraphics[width=\textwidth,height=\textheight,keepaspectratio]{images/compare_pop.png}
    \caption{Comparing modelled population with actual population by age groups of Kirbati from 1950 to 2020.}
    \label{fig:compare_group}
\end{figure}