\section{Model calibration and uncertainty propagation}
\label{calibration}
\textcolor{red}{REVISE THE SECTION BELOW WHEN FINALISED}
The model was calibrated using a Bayesian approach. In particular, we used the adaptive 
Metropolis algorithm introduced by Haario \textit{et al.} to sample parameters from their 
posterior distributions \cite{haario-2001}. For each country, we ran 8 independent Metropolis 
chains initialised using Latin Hypercube Sampling based on the parameter priors. 
We ran simulations for X hours per chain in order to achieve at least 50,000 iterations per 
chain. We discarded the first 25,000 iterations of each chain as burn-in and combined the 
samples of the 8 chains to project epidemic trajectories over time. 
For each modelled country, the epidemic projections presented in our analysis are associated with 1000 parameter sets 
randomly sampled from the posterior distributions obtained from MCMC sampling. 
The definitions of the prior distributions and the likelihood are detailed in the following sections.

\subsection{Parameters varied during calibration}

The parameters varied during calibration along with their associated prior distributions 
are listed in Table (\textcolor{red}{add ref to table here}) and indicated with the superscript \textsuperscript{c}.
We used uniform prior distributions for all calibrated parameters. The primary parameters varied during calibration are 
the unadjusted risk of transmission per contact ($\beta$), the IFR multiplier ($m_C$), the infection seeding times of 
each strain and the proportion of students on-site during ``Partially open'' periods. Note that the values of the random 
process $W_t$ described in Section \ref{random_process} are also treated like calibrated parameters by the MCMC. 
They are associated with an improper uniform prior distribution, whereas the auto-regressive component described in 
Equation \ref{eq:random_process} is incorporated in the posterior likelihood computation (Section \ref{likelihood}).

\subsection{Calibration targets}
\label{targets}
For each country model calibration is achieved by targeting two COVID-19 burden indicators: the 
reported number of COVID-19 deaths over time and country-level seroprevalence estimates. The following sections
presents the data used for calibration.
\subsubsection{COVID-19 deaths}
\textcolor{red}{NEED TO COMPLETE THIS}

\subsubsection{Seroprevalence estimates}
\label{sero}
\textcolor{red}{Use Angus' description here.}

\subsection{Likelihood definition}
\label{likelihood}
Let $d_w$ denote the rounded average daily number of COVID-19 deaths during week $w$, and $\hat{d_w}^\theta$ 
the associated predicted number of deaths according to the model with parameter set $\theta$. 

Let us denote $m$ the sample size associated with the seroprevalence survey selected from SeroTracker (Section \ref{sero}), 
and $k$ the number of seropositive individuals observed in the survey.
Let $\hat{\pi}^\theta$ denote the modelled proportion ever infected by the time the survey was conducted (using midpoint date) 
associated with the parameter set $\theta$. 
The likelihood was defined as follows:

\begin{equation}
    \label{eq:likelihood}
    \mathcal{L}(\theta) := f_{m,\hat{\pi}^\theta}(k) \times \prod_w g_r(d_w | \:\hat{d_w}^\theta) \quad ,
\end{equation}
where $f_{n,p}(.)$ is the probability mass function of a binomial distribution $\mathcal{B}(n,p)$, and 
$g_r(. | \mu)$ is the probability mass function of a negative binomial distribution with mean $\mu$ and 
overdispersion parameter $r$. The parameter $r$ is automatically estimated by the MCMC algorithm.

The likelihood described above represents the goodness of fit of a particular model parameterisation with regards to the targeted data. 
This quantity needs to be adjusted for the prior likelihood of the parameter set in order to compute the MCMC acceptance quantity $\mathcal{Q}(\theta)$.
As we used uniform priors for all the parameters, the inclusion of the individual parameters' priors in the acceptance quantity is not necessary. 
Indeed, their respective contributions would cancel out as the same quantity would appear in the numerator and the denominator of the 
MCMC acceptance quantity ratio. However, the auto-regressive relationship described in Equation \ref{eq:random_process}
must be accounted for as part of the combined prior likelihood of a parameter set. This is what prevents unrealistic fluctuations of the random process.
If $W^\theta$ represents the random process associated with the parameter set $\theta$, the overall MCMC acceptance quantity is obtained by:

\begin{equation}
    \label{eq:acc_qtt}
    \begin{split}
    \mathcal{Q}(\theta) & = \mathcal{L}(\theta) \times \prod_{i=1}^{n} z_{W^\theta_{i-1},\sigma}(W^\theta_i) \\
                        & = f_{m,\hat{\pi}^\theta}(k) \times \prod_w g_r(d_w | \:\hat{d_w}^\theta) \times \prod_{i=1}^{n} z_{W^\theta_{i-1},\sigma}(W^\theta_i) \quad ,
    \end{split}
\end{equation}
where $z_{\mu,\sigma}(.)$ represents the probability density function of the normal distribution $\mathcal{N}(\mu, \sigma)$, and $n$ is the number 
of random process updates. The standard deviation $\sigma$ is automatically estimated by the MCMC.

\subsection{Model coding and run time}
\textcolor{red}{NEED TO COMPLETE THIS. Maybe David can help here?}
Can probably find a better title for this section\dots

We used the \textit{summer} Python package (v 1.2.5) to implement the model \cite{summer2024}. This is a domain-specific library for 
compartmental epidemiological modelling enabling highly expressive programming and accelerated computation through Google’s \textit{jax} library

\subsubsection{Application Programming Interface (API)}
The model specification was done via a simple yet expressive Python API, 
while the numerical implementation was autogenerated by the \textit{summer} package at runtime. 
This specification is composable via stratification classes and other reusable components, 
such that the complexity of the software is kept to a minimum, reducing cognitive overhead and greatly reducing the possibility for error.

\subsubsection{Optimising compiler}
The \textit{summer} package uses the jax library as its computational backend, meaning that while the specification of models is done largely in Python, 
the model execution itself is transformed via an optimising compiler into fast native code.
This brings the model runtime from several seconds (for a naive implementation) to under 50ms per iteration, which was necessary to perform the computationally-intensive
calibration tasks described in Section \ref{calibration}.





