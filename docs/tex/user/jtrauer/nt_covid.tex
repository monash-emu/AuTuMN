\documentclass{article}
\author{
    Epidemiological Modelling Unit,
    \\ School of Public Health and Preventive Medicine,
    \\ Monash University
}
\usepackage{graphicx}
\usepackage{biblatex}
\usepackage{threeparttable}
\usepackage{tabularx}
\usepackage{hyperref}
\usepackage[left=2.5cm,right=2.5cm]{geometry}
% TeX code to create a custom command called \code to allow for the inclusion of inline code
\usepackage{xcolor}
\definecolor{light-gray}{gray}{0.9}  % Define the background colour
\newcommand{\code}[1]{\colorbox{light-gray}{\texttt{#1}}}  % Manual creation of an inline code command

\bibliography{../../references/emu_library}

\title{Modelling methods, Northern Territory application}

\begin{document}

\maketitle
\tableofcontents
\newpage
\begin{document}

\section{Model Structure}

\end{document}
\begin{figure}[ht]
    \includegraphics[width=\textwidth]{../../tex_descriptions/models/tuberculosis/model.png}
    \caption[Unstratified compartmental model structure.]{Unstratified compartmental model structure. \small S = susceptible, E: Early LTBI, L: Late LTBI, E = exposed, I = active, R = recovered/removed}
    \label{fig:seeiir}
\end{figure}
\section{Age stratification}
% Note that this will vary for every application, so will need to be edited - not sure of how best to manage this:
All compartments of the base compartmental structure were stratified by age into the following age bands:
\begin{itemize}
    \item 0 to 4 years
    \item 5 to 9 years
    \item 10 to 14 years
    \item 15 to 19 years
    \item 20 to 24 years
    \item 25 to 29 years
    \item 30 to 34 years
    \item 35 to 39 years
    \item 40 to 44 years
    \item 45 to 49 years
    \item 50 to 54 years
    \item 55 to 59 years
    \item 60 to 64 years
    \item 65 to 69 years
    \item 70 to 74 years
    \item 75 years and above
\end{itemize}
Demographic processes, including births, ageing and non-infection-related deaths 
are not simulated, given the timeframes considered in this simulation.

\section{Infectious seed}
The infectious seed value is assigned at the start of simulation process.
This value is subtracted from the total modelled population and assigned to the susceptible compartment, while all other compartments are assigned a starting value of zero.
This process is undertaken before age stratification is applied,
with the stratification process then splitting these values proportionately according to the starting age distribution of the population.

\subsection{Clinical stratification} \label{clin}
The age-stratified late exposed/incubation and both the early and late active disease compartments were further stratified into five ``clinical" categories: \textit{1)} asymptomatic, \textit{2)} symptomatic ambulatory, never detected, \textit{3)} symptomatic ambulatory, ever detected, \textit{4)} ever hospitalised, never critical, and \textit{5)} ever critically unwell (Figure \ref{fig:clinical_strat}).
The proportion of new infectious persons entering stratum 1 (asymptomatic) is age-dependent and constant over time. The proportion of symptomatic patients (strata 2 to 5) ever detected (strata 3 to 5) is set through a parameter that represents the time-varying proportion of all symptomatic patients who are ever detected (the case detection rate). Of those ever symptomatic (strata 2 to 5), an age-specific proportion is considered to be hospitalised (entering strata 4 or 5, constant over time). Of those hospitalised (entering strata 4 or 5), a fixed proportion was considered to be critically unwell (entering stratum 5, Figure \ref{fig:clinical_rationale}, not age-specific, constant over time).

\begin{figure}[ht]
    \includegraphics[width=\textwidth]{../covid_19/stratifications/covid_19_clinical_strat.pdf}
    \title{Illustration of the implementation of the clinical stratification.}
    \caption{\textbf{Illustration of the implementation of the clinical stratification.} Depth of pink/red shading indicates the infectiousness of the compartment. Typical parameter values represented, although the infectiousness of asymptomatic persons is varied in calibration.}
    \label{fig:clinical_strat}
\end{figure}

\begin{figure}[ht]
    \includegraphics[width=\textwidth]{../covid_19/stratifications/covid_19_clinical_rationale.pdf}
    \title{Illustration of the rationale for the clinical stratification.}
	\caption{\textbf{Illustration of the rationale for the clinical stratification.}}
    \label{fig:clinical_rationale}
\end{figure}

\subsection{Hospitalisation} \label{hosp}
For COVID-19 patients who are admitted to hospital, the sojourn time in the early and late active compartments is modified, superseding the default values of the sojourn times for these compartments. The point of admission to hospital is considered to be the transition from early to late active disease, such that the sojourn time in the late disease represents the period of time admitted to hospital. For patients admitted to ICU, admission to ICU occurs at this same transition point. For this group, the period of time hospitalised prior to ICU admission is estimated as a proportion of the early active period, such that the early active period represents both the period ambulatory in the community and the period in hospital prior to ICU admission.

\subsection{Infectiousness} \label{infect}
Asymptomatic persons are assumed to be less infectious per unit time active than symptomatic persons not undergoing case isolation (typically by around 50\%, although this is varied in calibration/uncertainty analysis). Infectiousness is also decreased for persons who have been detected to reflect case isolation, and for those admitted to hospital or ICU to reflect infection control procedures (by 80\% for all groups, parameter value consistent with typically used modelled values but not informed by empiric estimates). Presymptomatic individuals are assumed to have equivalent infectiousness to those with early active COVID-19.

\subsection{Application of COVID-19-related death}
Age-specific infection fatality rates (IFRs) were applied and distributed across strata 4 and 5, with no deaths typically applied to the first three strata. A ceiling of 50\% is set on the proportion of those admitted to ICU (entering stratum 5) who die. If the infection fatality rate is greater than this ceiling, the proportion of critically unwell persons dying was set to 50\%, with the remainder of the infection fatality rate then applied to the hospitalised proportion. Otherwise, if the infection fatality rate is less than half of the absolute proportion of persons critically unwell, the infection fatality rate is applied entirely through stratum 5 (such that the proportion of critically unwell persons dying in that age group becomes \textless 50\% and the proportion of stratum 4 dying is set to zero). In the event that the infection fatality rate for an age group is greater than the total proportion hospitalised (which is unusual, but could occur for the oldest age group under certain parameter configurations), the remaining deaths are assigned to the asymptomatic stratum. This approach was adopted for computational ease and is valid because the duration active for persons entering this stratum is the same as for the other non-hospitalised strata, such that the dynamics are identical to assigning the deaths to any of the first three strata. We used the age-specific IFRs previously estimated from age-specific death data from 45 countries and results from national-level seroprevalence surveys \cite{RN6}.

\begin{table}[ht]
\renewcommand{\baselinestretch}{1}
    \begin{tabular}{| p{2cm} | p{6.6cm} | p{2.5cm} | l | l |}
    	\hline
        \textbf{Clinical stratum} & \textbf{Stratum name} & \textbf{Pre-symptomatic} &\textbf{Early} & \textbf{Late}\\
        \hline
        1 & Asymptomatic & \cellcolor[HTML]{DC6464}\textcolor[HTML]{FFFFFF}{0.5} & 
        \cellcolor[HTML]{DC6464}\textcolor[HTML]{FFFFFF}{0.5} & \cellcolor[HTML]{DC6464}\textcolor[HTML]{FFFFFF}{0.5} \\
        2 & Symptomatic ambulatory never detected & \cellcolor[HTML]{C90000}\textcolor[HTML]{FFFFFF}{1} & \cellcolor[HTML]{C90000}\textcolor[HTML]{FFFFFF}{1} & \cellcolor[HTML]{C90000}\textcolor[HTML]{FFFFFF}{1} \\
        3 & Symptomatic ambulatory ever detected & \cellcolor[HTML]{C90000}\textcolor[HTML]{FFFFFF}{1} & \cellcolor[HTML]{C90000}\textcolor[HTML]{FFFFFF}{1} & \cellcolor[HTML]{F0BEBE}\textcolor[HTML]{FFFFFF}{0.2} \\
        4 & Hospitalised never critical & \cellcolor[HTML]{C90000}\textcolor[HTML]{FFFFFF}{1} & \cellcolor[HTML]{C90000}\textcolor[HTML]{FFFFFF}{1} & \cellcolor[HTML]{F0BEBE}\textcolor[HTML]{FFFFFF}{0.2} \\
        5 & Ever critically unwell & 
        \cellcolor[HTML]{C90000}\textcolor[HTML]{FFFFFF}{1} & \cellcolor[HTML]{C90000}\textcolor[HTML]{FFFFFF}{1} & \cellcolor[HTML]{F0BEBE}\textcolor[HTML]{FFFFFF}{0.2} \\
        \hline
    \end{tabular}
    	\title{Illustration of the relative infectiousness of disease compartments by clinical stratification and stage of infection.}
    \caption{\textbf{Illustration of the relative infectiousness of disease compartments by clinical stratification and stage of infection.} Typical parameter values displayed.}
    \label{tab:clinical}
\end{table}

\subsection{Modelling Variants of Concern (VoC)}
To consider the effects of VoC on infection dynamics and the vaccination programs, we explicitly simulated two competing strains to represent 1) the wild-type or ancestral virus, and 2) all VoC strains, where the VoC were assumed to be associated with increased transmissibility only, which is set to be calibrated from the model. Therefore, we do not differentiate between different variants and assume the single VoC strain that is modelled represents all currently circulating strains. Susceptible individuals can be infected with either the wild-type or VoC strain and infectious individuals contribute to the force of infection with their respective infecting strain only. VoC strains are seeded into the model such that one additional person per day is infected with the VoC strain for a duration of ten days, with the time that this ten-day period commences varied during model calibration.
% Note that it would be good to flesh out this strain section to be NT-specific
%\section{Vaccination history stratification}
% Referring to the immunity stratification as the vaccination stratification here, for the current application
History of vaccination is captured by stratifying all model compartments by vaccination status.
Two vaccination strata are included to represent those who have received at least two doses of a COVID-19 vaccine,
and those who have not.

%\section{Scaling of contact rates}
The base mixing matrix can be disaggregated into age-specific contacts that occur 
in the following four settings:
\begin{itemize}
    \item Home
    \item Other locations
    \item Work
    \item School
\end{itemize}
The contact rates in each of these settings are scaled as indicated in 
Table \ref{tab:location_scaling}.
The quantity indicated in the Table is considered as a proportion relative to
the baseline value of one.
This quantity is then squared to consider that reductions in mobility will affect
both the potential infector and infectee who will not visit the setting considered.


% \begin{table}[h]
%     \begin{threeparttable}
%     \begin{tabularx}{\textwidth}{| X | X |}
%         \hline
%         \textbf{Matrix contact setting} & \textbf{Data used to scale contacts} \\
%         \hline
%         Home & Not scaled \\
%         \hline
%         Other locations & Facebook ``tiles visited'' metric \tnote{a} \\
%         \hline
%         Work & Facebook ``tiles visited'' metric \tnote{a} \\
%         \hline
%         School & Historical timeline of school closures/re-openings in Bhutan \\
%         \hline
% 	\end{tabularx}
% 	\caption{\textbf{Scaling of mobility values by matrix setting.}}
% 	\label{tab:location_scaling}
%     \begin{tablenotes}
%         \item[a] Google mobility data not available for Bhutan.
%     \end{tablenotes}
%     \end{threeparttable}
% \end{table}

% \subsection{Estimation of microdistancing effect}
% In addition to the mobility considerations, we also consider that other
% behavioural changes independent of whether certain settings were visited
% may be relevant in Bhutan.
% This further scales the rates of contact at the work and other locations settings,
% with the effect squared as for the mobility considerations.

%\section{Calculation of outputs}

\subsection{Incidence}
Incidence is calculated as any transitions into the early active compartment (\textit{``I"}).

\subsection{Hospital occupancy}
Hospitalisation numbers are not reported in the case of Sri Lanka, because the approach to hospitalisation in the country differ considerably from that adopted in the other countries to which the AuTuMN model is applied. Therefore, our approach to simulating hospitalisations is not applicable to Sri Lanka.

\subsection{ICU occupancy}
This is calculated as all persons in the late active compartment in clinical stratum 5.

\subsection{Seropositive proportion}
This is calculated as the proportion of the population in the recovered (\textit{``R"}) compartment. Although very similar numerically to the attack rate, persons who died of COVID-19 are not included in the denominator.

\subsection{COVID-19-related mortality}
This is calculated as all transitions representing death, exiting the model. This is implemented as depletion of the late active compartment.

\subsection{Notifications}
Local case notifications are calculated as transitions from the early to the late active compartment for clinical strata 3 to 5.




\section{Parameters}
\subsection{Age-specific parameters}
% *** Note, I don't think this is correct - the IFR is now changed to CFR,
% while I haven't got to documenting how hospitalisation rates are calculated yet
Age-structured parameters are presented in Table \ref{tab:age_params}.

\begin{table}
    \begin{threeparttable}
    \begin{tabularx}{\textwidth}{| X | X | X | X | X |}
        \hline
        Age group (years) & Clinical fraction\tnote{a} & 
        Relative susceptibility to infection & Case fatality rate & 
        Proportion of symptomatic patients hospitalised \\
        \hline
        0 to 4 & 0.533 & 0.36 & $1\times10^{-5}$ & 0.011 \\
        \hline
        5 to 9 & 0.533 & 0.36 & $1\times10^{-5}$ & 0.011 \\
        \hline
        10 to 14 & 0.533 & 0.36 & $1\times10^{-5}$ & 0.0038 \\
        \hline
        15 to 19 & 0.533 & 1 & $1\times10^{-5}$ & 0.0038 \\
        \hline
        20 to 24 & 0.679 & 1 & $4\times10^{-5}$ & 0.060 \\
        \hline
        25 to 29 & 0.679 & 1 & $4\times10^{-5}$ & 0.060 \\
        \hline
        30 to 34 & 0.679 & 1 & $4\times10^{-4}$ & 0.0066 \\
        \hline
        35 to 39 & 0.679 & 1 & $4\times10^{-4}$ & 0.0066 \\
        \hline
        40 to 44 & 0.679 & 1 & $5\times10^{-4}$ & 0.0059 \\
        \hline
        45 to 49 & 0.679 & 1 & $5\times10^{4}$ & 0.0059 \\
        \hline
        50 to 54 & 0.679 & 1 & $2.6\times10^{-3}$ & 0.0077 \\
        \hline
        55 to 59 & 0.679 & 1 & $2.6\times10^{-3}$ & 0.0077 \\
        \hline
        60 to 64 & 0.803 & 1 & $1.12\times10^{-2}$ & 0.139 \\
        \hline
        65 to 69 & 0.803 & 1.41 & $4.93\times10^{-2}$ & 0.139 \\
        \hline
        70 to 74 & 0.803 & 1.41 & $4.93\times10^{-2}$ & 0.0357 \\
        \hline
        75 and above & 0.803 & 1.41 & $\times10^{-1}$\tnote{b} & 0.111 \\
        \hline
        Justification and source & 
        Table 1 of systematic review and meta-analysis with appropriate accounting for testing during the pre-symptomatic period\cite{sah-2021}. & 
        Conversion of odds ratios presented in Table S15 of Zhang et al. 2020 to relative risks using data presented in Table S14 of the same study\cite{zhang-2020-a}\tnote{c}. &
        Estimated from pooled analysis of data from 45 countries from Table S3 of O'Driscoll et al \cite{odriscoll-2021}. 
        Values consistent with previous estimates using serosurveys performed in Spain \cite{pollan-2020}. &
        Table S2 from Supplemental Materials to Nyberg et al. \cite{nyberg-2022} \\ 
        \hline
	\end{tabularx}
	\caption{Age-stratified parameters not varied during calibration, or varied through a common multiplier.}
	\label{tab:age_params}
    \begin{tablenotes}
        \item[a] Proportion of incident cases developing symptoms.
        \item[b] Weighted average of IFR estimates for 70 to 79 and 80 and above age groups.
        \item[c] Note the relative magnitude of these values are similar to those estimated by the analysis we use to estimate the age-specific clinical fraction.
    \end{tablenotes}
    \end{threeparttable}
\end{table}


\begin{table}[h]
    \begin{tabularx}{\textwidth}{| X | p{2.5cm} | X |}
    \hline
    \textbf{Parameter} & \textbf{Value} & \textbf{Rationale} \\
    \hline
    
Probability of transmission per contact & Calibrated  & Assumed \\ 
\hline
ISO3 code for source country for mixing matrix & GBR  & Assumed \\ 
\hline
Starting infectious seed & 2.0 persons & Assumed \\ 
\hline
Proportion of active period before isolation & 0.333  & Assumed \\ 
\hline
Relative infectiousness of asymptomatic persons & 0.5  & Assumed \\ 
\hline
Relative infectiousness of isolated cases & 0.2  & Assumed \\ 
\hline
Duration of booster effect & 90 days & Assumed \\ 
\hline
Reduction in transmission risk for boosted & 0.6  & Assumed \\ 
\hline
Reduction in transmission risk for primary course & 0.0  & Assumed \\
    \hline
    \end{tabularx}
	\caption{\textbf{Epidemiological fixed parameter values.}}
    \label{tab:fixed_params}
\end{table}

\begin{table}[h]
    \begin{tabularx}{\textwidth}{| X | p{2.5cm} | X |}
    \hline
    \textbf{Parameter} & \textbf{Value} & \textbf{Rationale} \\
    \hline
    
Distribution type for hospital stay & gamma  & Assumed \\ 
\hline
Mean hospital stay & 3.0 days & Assumed \\ 
\hline
Shape parameter for hospital stay & 5.0  & Assumed \\ 
\hline
Distribution type for ICU stay & gamma  & Assumed \\ 
\hline
Mean ICU stay & 4.7 days & Assumed \\ 
\hline
Shape parameter for ICU stay & 5.0  & Assumed \\ 
\hline
Proportion of hospitalised persons admitted to ICU & 0.08  & Assumed \\
    \hline
    \end{tabularx}
	\caption{\textbf{Output-related fixed parameter values.}}
    \label{tab:output_params}
\end{table}

\begin{table}[h]
    \begin{tabularx}{\textwidth}{| X | p{2.5cm} | X |}
    \hline
    \textbf{Parameter} & \textbf{Distribution type} & \textbf{Distribution parameters} \\
    \hline
    
Probability of transmission per contact & Uniform & turnips \\ 
\hline
Sojourns.latent.total time & Uniform & turnips \\ 
\hline
Sojourns.active.total time & Uniform & turnips \\ 
\hline
Starting infectious seed & Uniform & turnips \\ 
\hline
Mobility.microdistancing.behaviour.parameters.max effect & Uniform & turnips \\
    \hline
    \end{tabularx}
	\caption{\textbf{Output-related fixed parameter values.}}
    \label{tab:priors}
\end{table}

\newpage
\printbibliography

\end{document}
