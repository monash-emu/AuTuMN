\documentclass{article}
\author{
    Epidemiological Modelling Unit,
    \\ School of Public Health and Preventive Medicine,
    \\ Monash University
}
\usepackage{graphicx}
\usepackage[style=nature, backend=biber]{biblatex}
\usepackage{threeparttable}
\usepackage{tabularx}
\usepackage[left=2.5cm,right=2.5cm,bottom=2.5cm]{geometry}
\usepackage[singlelinecheck=false,justification=justified,labelfont=bf]{caption}
\setlength{\belowcaptionskip}{-10pt}

\addbibresource{../../references/emu_library.bib}
\title{Modelling methods, Kiribati application}

\begin{document}
\linespread{1.25}

\maketitle
\tableofcontents
\newpage
\begin{document}

\section{Model Structure}

\end{document}

\section{Demographics}
\subsection{Crude birth rate and death rate}
Births are modelled using time-variant crude birth rates that are multiplied by the modelled population 
size to determine the number of newborn individuals entering the youngest age category (0 - 4) at each time. A time-variant 
and age-specific rate of non-TB-related mortality applies to all model compartments to simulate 
deaths from other causes than TB. We use estimates from the UN Population Division\footnote{https://population.un.org/wpp/Download/Standard/Mortality/} to inform the 
birth and mortality rates.
We also apply additional death rates to the compartment I to reflect mortality induced by TB 
disease.
\begin{table}[!htp]
    \caption{\textbf{Model Parameters}}
    \label{tab:parameter}
    \begin{tabularx}{\textwidth}{ X  X  c }
        \hline
        \textbf{Parameter} & \textbf{Value} & \textbf{Evidence} \\
        \hline
        \textbf{Population characteristics} & & \\
        
Probability of transmission per contact & Calibrated  & Assumed \\ 
\hline
ISO3 code for source country for mixing matrix & GBR  & Assumed \\ 
\hline
Starting infectious seed & 2.0 persons & Assumed \\ 
\hline
Proportion of active period before isolation & 0.333  & Assumed \\ 
\hline
Relative infectiousness of asymptomatic persons & 0.5  & Assumed \\ 
\hline
Relative infectiousness of isolated cases & 0.2  & Assumed \\ 
\hline
Duration of booster effect & 90 days & Assumed \\ 
\hline
Reduction in transmission risk for boosted & 0.6  & Assumed \\ 
\hline
Reduction in transmission risk for primary course & 0.0  & Assumed \\
        Crude birth rate  & Time-variant & UN Population Divison \\
        Universal death rate & Time-variant & UN Population Divison \\
        \hline
        \textbf{\emph{M.tb} infection and TB} \\
        Rate of stabilization from early to late latency &    
        \begin{minipage}[t]{0.3\textwidth}
            Age 0-4: 4.4 per year \newline
            Age 5-14: 4.4 per year \newline
            Age 15+: 2 per year \newline
        \end{minipage}
        & 
        \begin{minipage}[t]{0.4\textwidth}
            Table 1, Parameter estimates. Calibration issued from the survival likelihood maximisation applied 
            to the merged dataset (Victoria and Amsterdam data), Ragonnet et al.\cite{ragonnet-2017}
        \end{minipage} \\
        Rate of rapid progression to active TB & 
        \begin{minipage}[t]{0.3\textwidth}
            Age 0-4: 2.4 per year \newline
            Age 5-14: 2 per year \newline
            Age 15+: 0.1 per year \newline
        \end{minipage}
        & 
        \begin{minipage}[t]{0.4\textwidth}
            Table 1, Parameter estimates. Calibration issued from the survival likelihood maximisation applied 
            to the merged dataset (Victoria and Amsterdam data) \cite{ragonnet-2017}
        \end{minipage}  \\
        Rate of late reactivation & 
        \begin{minipage}[t]{0.3\textwidth}
            Age 0-4: 7e\textsuperscript{-9} per year \newline
            Age 5-14: 2.3e\textsuperscript{-3} per year  \newline
            Age 15+: 1.2e\textsuperscript{-3} per year \newline
        \end{minipage}
        &
        \begin{minipage}[t]{0.4\textwidth}
        Table 1, Parameter estimates. Calibration issued from the survival likelihood maximisation applied 
        to the merged dataset (Victoria and Amsterdam data), \cite{ragonnet-2017}
        \end{minipage} \\
        Relative risk of reinfection while latently infected (ref. infection-naive) & 0.2 &
        \begin{minipage}[t]{0.4\textwidth}
        The weighted mean adjusted incidence rate of tuberculosis in the LTBI and UI 
        groups attributable to reinfection was 13.5 per 1000 person-years
        (95\% confidence interval [CI]: 5.0–26.2 per 1000 person-years) and that 
        attributable to primary infection was 60.1 per 1000 person-years 
        (95\% CI: 38.6–87.4 per 1000 person-years). \cite{andrews-2012}
        \end{minipage} \\
        Relative risk of reinfection after recovery (ref. infection-naive) & 0.2 & \begin{minipage}[t]{0.4\textwidth}
        The adjusted IRR for tuberculosis in the LTBI group compared with the UI group was 0.21. \cite{andrews-2012} 
        \end{minipage} \\
        \hline
	\end{tabularx}
\end{table}

\begin{figure}[!htp]
   \includegraphics[width=\textwidth,keepaspectratio]{images/cbr.png}
    \caption{The crude birth rate of Kiribati from 1950 to 2020.}
    \label{fig:cbr}
\end{figure}

\begin{figure}[!htp]
   \centerline{\includegraphics[width=\textwidth,keepaspectratio]{images/cdr.png}}
    \caption{The death rate of Kiribati from 1950 to 2020, stratified by age group.}
    \label{fig:cdr}
\end{figure}

\subsection{Comparing modelled population with actual population}
\begin{figure}[!htp]
    \centerline{\includegraphics[width=\textwidth,keepaspectratio]{images/modelled_total.png}}
    \caption{Comparing modelled population with actual population of Kirbati from 1950 to 2020. The red dots represent the actual population size of Kiribati,
     while the blue line represents the modelled population size}
    \label{fig:modelled_total}
\end{figure}

\begin{figure}[htp]
    \centerline{\includegraphics[width=\textwidth,keepaspectratio]{images/compare_pop.png}}
    \caption{Comparing modelled population with actual population by age groups of Kirbati from 1950 to 2020.}
    \label{fig:compare_group}
\end{figure}

\newpage    
\printbibliography
\end{document}
