\section{Case detection and isolation} \label{cdr}

\subsection{Determining the proportion of cases detected}
We calculate a time-varying case detection rate, being the proportion of all symptomatic cases (the second and third clinical strata) that are detected (the third clinical stratum only).
This proportion is informed by the number of tests performed using the following formula:

\[CDR(time)=1-(1-floor)\times e^{-shape \times tests(time)}\]

$time$ is the calendar date and $tests(time)$ is the number of tests per capita done on that date. To determine the value of the shape parameter, we solve this equation based on the assumption that a certain daily testing rate $tests(t)$ is associated with a certain $CDR(t)$.
$floor$ is the minimum case detection rate possible, which would theoretically occur when zero tests are conducted.
Solving for $shape$ yields:

\[shape = \frac{-log(\frac{1 - CDR(t)}{1 - floor})}{tests(t)}\]

That is, if it is assumed that a certain daily per capita testing rate is associated with a certain proportion of symptomatic cases detected, we can determine $shape$.
As this relationship is not well understood and unlikely to be consistent across all settings, we vary the $CDR$ that is associated with a certain per capita testing rate during calibration.
This approach allows us to both vary the $CDR(\cdot)$ relationship through calibration,
while also varying the specific $CDR(time)$ to reflect historical changes in testing capacity with time.

\subsection{Isolation of detected cases}
As described in the clinical stratification section above, as infected persons progress from the early to the late stage of active COVID-19, infectiousness is reduced for those detected to reflect case isolation.
