\section{Vaccination history stratification}
% Referring to the immunity stratification as the vaccination stratification here, for the current application
History of vaccination is captured by stratifying all model compartments by vaccination status.
Two vaccination strata are included to represent those who have received at least two doses of a COVID-19 vaccine,
and those who have not.
The following are not considered in this approach to simulating vaccination:
\begin{itemize}
    \item Any effect of receiving a single dose of vaccine
    \item Waning of vaccine-induced immunity
    \item Any additional effect from receiving additional vaccine doses following the second dose
\end{itemize}
The effect of vaccination on transmission is to partially reduce the rate of infection for all persons at-risk of infection in the vaccinated stratum.
This includes both fully susceptible (never previously infected) persons,
as well as recovered persons who are at risk of reinfection.
Emerging variants of concern (VoCs) may escape this immunity, as described further below.
% To illustrate this further, the next steps would be to display the population distribution by stratum,
% and/or to explain the implementation of the dynamic proportions code
