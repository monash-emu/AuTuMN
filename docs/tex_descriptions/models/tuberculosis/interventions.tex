\subsection{Population-based screening and treatment of LTBI}
Mass LTBI screening and treatment is implemented as part of the intervention conducted in South Tarawa in 
2022. This is modelled by making latently infected individuals (from E and L) transition to the 
recovered compartment (R). The rate associated with these flows is obtained by multiplying the LTBI 
screening rate with the sensitivity of the LTBI test employed and the individual-level efficacy of 
preventive treatment. The LTBI screening rate is implemented as a time-variant parameter that is 
stratified by location
\subsection{Active case finding}
Active case finding (ACF) is implemented to simulate the interventions linked to the detection of 
individuals with active TB implemented in South Tarawa. This is modelled by 
implementing an additional transition flow from compartment I to compartment T. The rate associated 
with this flow is obtained by multiplying the location-specific ACF screening rate with the sensitivity 
of the detection algorithm used for the ACF intervention. The ACF screening rate is implemented as a 
time-variant parameter.
\subsection{Calculation of the screening rates}
To simulate the interventions, we apply a positive rate of ACF and/or LTBI screening over the 
intervention periods. The screening rates are determined such that the modelled total proportion of the 
population screened corresponds to the true population proportion screened. The screening rate is set 
equal to \(-log(1 - coverage)\) for the year during which the intervention is implemented, where 
\(coverage\) is the total proportion of the population screened by the intervention. 