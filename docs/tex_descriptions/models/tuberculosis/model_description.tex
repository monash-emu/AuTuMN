% This file describes the general structure of some of our sm_sir models.
% Note that it is possible to construct models in many different ways using the sm_sir code.
% Therefore, this description should be used with caution, because it only describes one possible model configuration.

\section{Model Structure}

\subsection{General features}

We use a deterministic compartmental model including six types of compartments that represent different states of infection and disease. 
The model uses the same conceptual approach and similar assumptions to previously published models. 
Here we describe the model structure before applying any stratification.
\subsection{Compartments}
Model compartments represent sequential progressions through the processes of infection with, progression through and recovery from the phases of Tuberculosis.
Reinfection is permitted in our model code structure, which is represented as transition from the recovered compartments back to the first infected compartment.
The following compartments are implemented:
\begin{itemize}
    \item Susceptible
    \begin{itemize}
        \item A susceptible compartment (S) is used to represent individuals who have never been infected with Mycobacterium tuberculosis (M.tb)
    \end{itemize}
    \item Latent
    \begin{itemize}
        \item Latent TB infection (LTBI) is modelled using two successive compartments: early latent (E) and late latent (L) to capture the declining risk of disease progression over time from infection
    \end{itemize}
    \item Infectious
    \begin{itemize}
        \item The active disease compartment (I) represents individuals who have progressed to the active stage of TB disease
    \end{itemize}
    \item Treatment
    \begin{itemize}
        \item All diseased individuals who are detected are assumed to be started on treatment. Treatment may result in cure (progression to R), relapse (return to I) or death
    \end{itemize}
    \item Recovered or Removed
    \begin{itemize}
        \item Persons recovered or removed/died from Tuberculosis during the model simulation period.
    \end{itemize}
\end{itemize}
