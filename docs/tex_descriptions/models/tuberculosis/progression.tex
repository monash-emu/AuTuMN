\subsection{Progression from latent to active TB}
We use the estimates reported in Ragonnet et al. to inform the modelled dynamics of activation from 
latent to active TB. These parameters vary by age, and a multiplier is used to 
incorporate uncertainty around the progression rates
\subsection{Effect of diabetes}
The model is not stratified by diabetes status. Instead, we model the effect of diabetes type 2 by 
increasing the rates of progression from latent to active TB using age-specific multipliers. For each 
age group, the value of the diabetes-effect multiplier depends on the age-specific proportion of 
diabetic individuals and the relative rate of TB reactivation for diabetic individuals compared to non-diabetic individuals.
\subsection{Natural history flows}
We use the estimates reported in Ragonnet et al. to model the rate of TB mortality in the absence of 
treatment and the rate of self-recovery. We use different rates of untreated TB mortality and self-recovery for smear-positive TB compared to smear-negative TB. The TB mortality and self-recovery 
rates associated with extrapulmonary TB are assumed to be the same as those of smear-negative TB.
\subsection{Passive case detection of active TB}
The detection rate is defined as the rate of progression from the active disease to the treatment 
compartment, as all detected individuals are assumed to be started on treatment at diagnosis in our 
model. This rate is calculated by multiplying the screening rate with the diagnostic test sensitivity. 
The screening rate can be interpreted as the reciprocal of the average time that diseased individuals 
take to seek care. The diagnostic sensitivity varies according to the organ status to reflect the relative 
differences in the difficulty to diagnose smear-negative TB and extrapulmonary TB, as compared to 
smear-positive TB.
We use a time-variant function to model the screening rate in order to capture detection improvements 
over time.
\subsection{Treatment outcomes}
Treated individuals can experience three different treatment outcomes: treatment success, relapse or 
death. The rate of treatment-induced recovery \phi{} is set to the reciprocal of the duration of a 
completed treatment course. We then use the observed treatment success proportion (often referred to 
as “treatment success rate”) as model input. In our model, it is calculated from \(TSR = \frac{\phi}{\phi + \rho + \mu_{\tau} + \mu^{'}}\){}, 
where \rho{} is the relapse rate, \({\mu_{\tau}{}}\) is the excess mortality rate of individuals on TB treatment, and \mu is the 
non-TB-related mortality rate. Finally, we calculate the respective values of {\rho}{} and \({\mu_{\tau}{}}\) using the 
observed proportion of deaths among all negative treatment outcomes, denoted \pi{}. We have \({\pi = \frac{\mu_{\tau} +\mu}{\rho + \mu_{\tau} + \mu}}\){} that we inject into the \({TSR}\) equation.

